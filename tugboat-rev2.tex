% (find-dednat6 "tugboat-rev2.tex")
% (defun c () (interactive) (find-dednat6sh "lualatex -record tugboat-rev2.tex" 1 ee-end))
% (defun d () (interactive) (find-pdf-page "~/dednat6/tugboat-rev2.pdf"))
% (defun e () (interactive) (find-dednat6            "tugboat-rev2.tex"))
% (defun u () (interactive) (find-latex-upload-links "tugboat-rev2"))
% (find-pdf-page   "~/dednat6/tugboat-rev2.pdf")
% (find-pdf-tex  t "~/dednat6/tugboat-rev2.pdf")
% (find-sh0 "cp -v  ~/dednat6/tugboat-rev2.pdf /tmp/")
% (find-sh0 "cp -v  ~/dednat6/tugboat-rev2.pdf /tmp/pen/")
%   file:///home/edrx/dednat6/tugboat-rev2.pdf
%                 file:///tmp/tugboat-rev2.pdf
%             file:///tmp/pen/tugboat-rev2.pdf
% http://angg.twu.net/dednat6/tugboat-rev2.pdf

% This is an article about Dednat6 for TUGBoat, called:
%
%   Dednat6: An extensible (semi-)preprocessor for LuaLaTeX
%     that understands diagrams in ASCII art
%
% This revised version incorporates the changes that Karl Berry made
% to the article to get rid of bad line breaks, but it says
% "preliminary version" and the page numbers are placeholder-ish.

% «.title»			(to "title")
% «.prehistory»			(to "prehistory")
% «.dednat.icn»			(to "dednat.icn")
% «.dednat.lua»			(to "dednat.lua")
% «.2D-low-level»		(to "2D-low-level")
% «.2D-code»			(to "2D-code")
% «.semi-preprocessors»		(to "semi-preprocessors")
%   «.heads-and-blocks»		(to "heads-and-blocks")
%   «.implementation-of-pu»	(to "implementation-of-pu")
% «.creating-new-heads»		(to "creating-new-heads")
% «.REPL»			(to "REPL")
% «.availability»		(to "availability")
% «.references»			(to "references")

%\documentclass[final]{ltugboat}
\documentclass{ltugboat}
\usepackage{microtype}
\usepackage{amsmath}
\usepackage{amsfonts}
\usepackage{amssymb}
%
% (find-dn6 "preamble6.lua" "preamble0")
\usepackage{proof}                    % For derivation trees ("%:" lines)
\input diagxy                         % For 2D diagrams ("%D" lines)
%
\catcode`\^^J=10                      % (find-es "luatex" "spurious-omega")
\directlua{dofile "dednat6load.lua"}  % (find-LATEX "dednat6load.lua")
%\usepackage{edrx15}                  % (find-angg "LATEX/edrx15.sty")
%\input edrxaccents.tex               % (find-angg "LATEX/edrxaccents.tex")
%\input edrxchars.tex                 % (find-LATEX "edrxchars.tex")
\usepackage{url}

\begin{document}

\catcode`\^^J=10                      % (find-es "luatex" "spurious-omega")
\catcode`\^^O=13 \def{{\color{red}*}}


% \co: a low-level way to typeset code; a poor man's "\verb"
% (find-es "tex" "co")
\def\co#1{{%
  \def\\{\char92}%
  \tt#1%
  }}

%L addabbrevs("->", "\\to ")

% http://linorg.usp.br/CTAN/macros/latex/contrib/tugboat/ltubguid.pdf
% http://angg.twu.net/math-b.html


%  _____ _ _   _      
% |_   _(_) |_| | ___ 
%   | | | | __| |/ _ \
%   | | | | |_| |  __/
%   |_| |_|\__|_|\___|
%                     
% «title» (to ".title")
% (tubp 1)

\title{Dednat6: An extensible (semi-)preprocessor for Lua\LaTeX\
  that understands diagrams in~\ASCII\ art}
\author{Eduardo Ochs}
\EDITORnoaddress
\netaddress{eduardoochs (at) gmail dot com}
\personalURL{http://angg.twu.net/dednat6.html} \maketitle

%                 _     _     _                   
%  _ __  _ __ ___| |__ (_)___| |_ ___  _ __ _   _ 
% | '_ \| '__/ _ \ '_ \| / __| __/ _ \| '__| | | |
% | |_) | | |  __/ | | | \__ \ || (_) | |  | |_| |
% | .__/|_|  \___|_| |_|_|___/\__\___/|_|   \__, |
% |_|                                       |___/ 
%
% «prehistory» (to ".prehistory")
% «dednat.icn» (to ".dednat.icn")
% (tugp 2)
% (tubp 1)
\section{Prehistory}
\label{prehistory}
\label{dednat.icn}

Many, many years ago, when I was writing my master's thesis, I
realized that I was typesetting too many natural deduction trees, and
that this was driving me mad. The code (in \texttt{proof.sty}) for a small tree
like this one
%
%:  [a]^1  a->b
%:  -----------
%:       b       b->c
%:       ------------
%:            c
%:          ----1
%:          a->c
%:
%:          ^a->c
%:
\pu
$$\ded{a->c}$$
%
was this:

\begin{verbatim}
\infer[{1}]{ a\to c }{
 \infer[{}]{ c }{
  \infer[{}]{ b }{
   [a]^1 &
   a\to b } &
  b\to c } } }
\end{verbatim}

This was somewhat manageable, but the code for bigger trees was very hard
to understand and to debug. I started to add 2D representations of the
typeset trees above the code, and I defined a macro \co{\\defded} to
let me define the code for several trees at once, and a macro
\co{\\ded} to invoke that code later:

\begin{verbatim}
%  [a]^1  a->b
%  -----------
%       b       b->c
%       ------------
%            c
%          ----1
%          a->c
%
%          ^a->c
%
\defded{a->c}{
 \infer[{1}]{ a\to c }{
  \infer[{}]{ c }{
   \infer[{}]{ b }{
    [a]^1 &
    a\to b } &
   b\to c } } }
%
$$\ded{a->c}$$
\end{verbatim}


Then I realized that if I made the syntax of my 2D representations a
bit more rigid, I could write a preprocessor that would understand
them directly, and write all the `\co{\\defded}'s itself to an
auxiliary file. If a file \co{foo.tex} had this (note: I will omit
all header and footer code, like \co{\\begin\{document\}} and
\co{\\end\{document\}}, from the examples),

\begin{verbatim}
\input foo.dnt

%:  [a]^1  a->b
%:  -----------
%:       b       b->c
%:       ------------
%:            c
%:          ----1
%:          a->c
%:
%:          ^a->c

$$\ded{a->c}$$
\end{verbatim}
%
then I just had to run ``\co{dednat.icn foo.tex;latex foo.tex}''
instead of ``\co{latex foo.tex}''.


%      _          _             _         _             
%   __| | ___  __| |_ __   __ _| |_      | |_   _  __ _ 
%  / _` |/ _ \/ _` | '_ \ / _` | __|     | | | | |/ _` |
% | (_| |  __/ (_| | | | | (_| | |_   _  | | |_| | (_| |
%  \__,_|\___|\__,_|_| |_|\__,_|\__| (_) |_|\__,_|\__,_|
%
% «dednat.lua» (to ".dednat.lua")
\section{\texttt{dednat.lua}}
\label{dednat.lua}

A few years after that, I learned Lua, fell in love with it, and ported
dednat.icn from Icon\Dash which was a {\sl compiled} language\Dash to
Lua.

The first novel feature in \co{dednat.lua} was a way to run arbitrary Lua
code from the \co{.tex} file being preprocessed, and so extend the
preprocessor dynamically. \co{dednat.lua} treated blocks of lines starting
with `\co{\%:}' as specifications of trees, and blocks of lines
starting with `\co{\%L}' as Lua code. More precisely, the initial set
of {\sl heads} was \co{\{"\%:", "\%L", "\%D"\}}, and \co{dednat.lua}
processed each block of contiguous lines starting with the same head
in a way that depended on the head.

The second novel feature in \co{dednat.lua} was a way to generate code for
categorical diagrams, or ``2D diagrams'' for short, automatically,
analogous to what we did for trees. I wanted to make the preprocessor write
the `\co{\\defdiag}'s seen here itself:

\begin{verbatim}
%   LA <-| A
%   |      |
%   v      v
%   B |-> RB
%
\defdiag{adj_L-|R}{
  \morphism(0,0)/<-|/<400,0>[LA`A;]
  \morphism(0,0)/->/<0,-400>[LA`B;]
  \morphism(400,0)/->/<0,-400>[A`RB;]
  \morphism(0,-400)/|->/<400,0>[B`RB;]
}
$$\diag{adj_L-|R}$$
\end{verbatim}
%
where `\co{\\morphism}' is the main macro in \co{diagxy}, Michael
Barr's front-end for \Xy-pic.

After months of experimentation I arrived at a good syntax for 2D
diagrams. This code:

\begin{verbatim}
%D diagram adj_L-|R
%D 2Dx     100    +25
%D 2D  100 LA <-| A
%D 2D      |      |
%D 2D      |      |
%D 2D      v      v
%D 2D  +25 B |-> RB
%D 2D
%D (( LA A <-|
%D    LA B -> A RB ->
%D    B RB |->
%D ))
%D enddiagram
%D
$$\diag{adj_L-|R}$$
\end{verbatim}
%
generates this:
%
%D diagram adj_L-|R
%D 2Dx     100    +25
%D 2D  100 LA <-| A
%D 2D      |      |
%D 2D      |      |
%D 2D      v      v
%D 2D  +25 B |-> RB
%D 2D
%D (( LA A <-|
%D    LA B -> A RB ->
%D    B RB |->
%D ))
%D enddiagram
%D
$$\pu
  \diag{adj_L-|R}
$$

The lines with `\co{\%D 2Dx}' and `\co{\%D 2D}' define a grid with
coordinates and nodes, and the lines between `\co{\%D ((}' and
`\co{\%D ))}' connect these nodes with arrows.



%  ____  ____      _                    _                _ 
% |___ \|  _ \ _  | | _____      __    | | _____   _____| |
%   __) | | | (_) | |/ _ \ \ /\ / /____| |/ _ \ \ / / _ \ |
%  / __/| |_| |_  | | (_) \ V  V /_____| |  __/\ V /  __/ |
% |_____|____/(_) |_|\___/ \_/\_/      |_|\___| \_/ \___|_|
%
% «2D-low-level» (to ".2D-low-level")
\subsection{A Forth-based language for 2D diagrams\Dash low-level ideas}
\label{2D-low-level}

The article ``Bootstrapping a Forth in 40 lines of Lua code'' [1]
describes how a Forth-like language can be reduced to a minimal
extensible core, and bootstrapped from it. The most basic feature in
[1] is ``words that eat text''; the fact that Forth
is a stack-based language is secondary \Dash stacks are added later. The
code for `\co{\%D}'-lines is based on [1].

A ``Forth'' \Dash actually the ``outer interpreter'' of
a Forth, but let's call it simply a ``Forth'' \Dash works on one line of
input at a time, reads each ``word'' in it and executes it as soon as
it is read. A ``word'' is any sequence of one of more non-whitespace
characters, and an input line is made of words separated by
whitespace. The ``outer interpreter'' of Forth does essentially this
on each line, in pseudocode:

\begin{verbatim}
while true do
  word = getword()
  if not word then break end
  execute(word)
end
\end{verbatim}

Note that \co{word} is a global variable. The current input line is
stored in \co{subj} and the current position of the parser is stored
in \co{pos}; \co{subj} and \co{pos} are also global variables \Dash
which means the \co{execute(word)} can change them!

The function \co{getword()} parses whitespace in \co{subj} starting at
\co{pos}, then parses a word and returns it, and advances \co{pos} to
the position after that word. There is a similar function called
\co{getrestofline()} that returns all the rest of the line from
\co{pos} onwards, and advances \co{pos} to the end of the line.

One of the simplest Forth words is `\co{\#}' (``comment''). It is
defined as:

\begin{verbatim}
forths["#"] = function ()
    getrestofline()
  end
\end{verbatim}

It simply runs \co{getrestofline()}, discards its return value, and
returns. We say that \co{\#} ``eats the rest of the line''.

In a ``real'' Forth we can define words using `\co{:}' and `\co{;}',
like this:

\begin{verbatim}
  : SQUARE DUP * ;
\end{verbatim}
%
but the Forth-based language in \co{dednat.lua} is so minimalistic that we
don't have `\co{:}' and `\co{;}' \Dash we define words by storing their
Lua code in the table \co{forths}.



%  ____  ____                    _      
% |___ \|  _ \ _    ___ ___   __| | ___ 
%   __) | | | (_)  / __/ _ \ / _` |/ _ \
%  / __/| |_| |_  | (_| (_) | (_| |  __/
% |_____|____/(_)  \___\___/ \__,_|\___|
%                                       
% «2D-code» (to ".2D-code")
\subsection{A Forth-based language for 2D diagrams \Dash code for diagrams}
\label{2D-code}

Let's look at an example. This code
%
% (find-dn6 "diagforth.lua" "2D-and-2Dx")
% (find-dn6 "diagstacks.lua")

\begin{verbatim}
%D diagram T:F->G
%D 2Dx     100 +20 +20
%D 2D  100     A
%D 2D         /|\
%D 2D        v v v
%D 2D  +30 FA --> GA
%D 2D
%D (( A FA |-> A GA |->
%D    FA GA -> .plabel= b TA
%D    A FA GA midpoint -->
%D ))
%D enddiagram
%D
$$\diag{T:F->G}$$
\end{verbatim}
%
yields this:
%
%D diagram T:F->G
%D 2Dx     100 +20 +20
%L print("xs:"); print(xs)
%D 2D  100     A
%D 2D         /|\
%D 2D        v v v
%D 2D  +30 FA --> GA
%L print("nodes:"); print(nodes)
%D 2D
%D (( A FA |-> A GA |->
%D    FA GA -> .plabel= b TA
%D    A FA GA midpoint -->
%L print("ds:"); print(ds)
%D ))
%L print("arrows:"); print(arrows)
%D enddiagram
%D
$$\pu
  \diag{T:F->G}
$$

The word \co{diagram} eats a word \Dash the name of the diagram \Dash and
sets \co{diagramname} to it. The word \co{2Dx} eats the rest of the
line, and uses it to attribute $x$-coordinates to some columns. The
word \co{2D} also eats the rest of the line; when it is followed by
$nnn$ or $+nnn$ that number gives the $y$-coordinate of that line, and
the words that intersect a point that has both an $x$-coordinate and a
$y$-coordinate become {\sl nodes}. When a \co{2D} is not followed by
an $nnn$ or $+nnn$ then this is a line without a $y$-coordinate, and
it is ignored.

In a sequence like ``\co{A FA |->}'', both \co{A} and \co{FA} put nodes
on the stack, and \co{|->} creates an arrow joining the two nodes on
the top of the stack, without dropping the nodes from the stack. In a
sequence like ``\co{FA GA midpoint}'' the \co{midpoint} creates a
phantom node halfway between the two nodes on the top of the stack,
drops (pops) them and pushes the phantom node in their place. The word
\co{.plabel=} eats two words, a {\sl placement} and a {\sl label}, and
modifies the arrow at the top of the stack by setting the arrow's
label and placement attributes with them. The word `\co{((}' remembers
the depth of the stack \Dash 42, say \Dash and the word `\co{))}' pops
elements from the top of the stack; if the depth at `\co{))}' is 200
then `\co{))}' pops $200-42$ elements to make the depth become 42
again.

The word \co{enddiagram} defines a diagram with the name stored in
\co{diagramname}; each arrow that was created, even the ones that were
dropped from the stack, becomes a call to \co{\\morphism} \Dash the main
macro in \co{diagxy} \Dash in the body of the diagram.

A good way to understand in detail how everything works is to inspect
the data structures. Let's modify the code of the example to add some
`\co{print}'s in `\co{\%L}'-lines in the middle of the
`\co{\%D}'-code:

\begin{verbatim}
%D diagram T:F->G
%D 2Dx     100 +20 +20
%L print("xs:"); print(xs)
%D 2D  100     A
%D 2D         /|\
%D 2D        v v v
%D 2D  +30 FA --> GA
%L print("nodes:"); print(nodes)
%D 2D
%D (( A FA |-> A GA |->
%D    FA GA -> .plabel= b TA
%D    A FA GA midpoint -->
%L print("ds:"); print(ds)
%D ))
%L print("arrows:"); print(arrows)
%D enddiagram
\end{verbatim}

The preprocessor outputs this on stdout:

\begin{verbatim}[\footnotesize]
xs:
{12=100, 16=120, 20=140}
nodes:
{ 1={"noden"=1, "tag"="A", "x"=120, "y"=100},
  2={"noden"=2, "tag"="FA", "x"=100, "y"=130},
  3={"noden"=3, "tag"="-->", "x"=120, "y"=130},
  4={"noden"=4, "tag"="GA", "x"=140, "y"=130},
  "-->"={"noden"=3, "tag"="-->", "x"=120, "y"=130},
  "A"={"noden"=1, "tag"="A", "x"=120, "y"=100},
  "FA"={"noden"=2, "tag"="FA", "x"=100, "y"=130},
  "GA"={"noden"=4, "tag"="GA", "x"=140, "y"=130}
}
ds:
12={"arrown"=4, "from"=1, "shape"="-->", "to"=5}
11={"TeX"="\\phantom{O}", "noden"=5, "x"=120,
    "y"=130}
10={"noden"=1, "tag"="A", "x"=120, "y"=100}
9={"arrown"=3, "from"=2, "label"="TA",
   "placement"="b", "shape"="->", "to"=4}
8={"noden"=4, "tag"="GA", "x"=140, "y"=130}
7={"noden"=2, "tag"="FA", "x"=100, "y"=130}
6={"arrown"=2, "from"=1, "shape"="|->", "to"=4}
5={"noden"=4, "tag"="GA", "x"=140, "y"=130}
4={"noden"=1, "tag"="A", "x"=120, "y"=100}
3={"arrown"=1, "from"=1, "shape"="|->", "to"=2}
2={"noden"=2, "tag"="FA", "x"=100, "y"=130}
1={"noden"=1, "tag"="A", "x"=120, "y"=100}
arrows:
{ 1={"arrown"=1, "from"=1, "shape"="|->", "to"=2},
  2={"arrown"=2, "from"=1, "shape"="|->", "to"=4},
  3={"arrown"=3, "from"=2, "label"="TA",
     "placement"="b", "shape"="->", "to"=4},
  4={"arrown"=4, "from"=1, "shape"="-->", "to"=5}
}
\end{verbatim}



%  ____                 _                                               
% / ___|  ___ _ __ ___ (_)      _ __  _ __ ___ _ __  _ __ ___   ___ ___ 
% \___ \ / _ \ '_ ` _ \| |_____| '_ \| '__/ _ \ '_ \| '__/ _ \ / __/ __|
%  ___) |  __/ | | | | | |_____| |_) | | |  __/ |_) | | | (_) | (__\__ \
% |____/ \___|_| |_| |_|_|     | .__/|_|  \___| .__/|_|  \___/ \___|___/
%                              |_|            |_|                       
%
% «semi-preprocessors» (to ".semi-preprocessors")
\section{Semi-preprocessors}

\co{dednat.icn}, \co{dednat.lua} and all its successors until \co{dednat5.lua} were
preprocessors in the usual sense \Dash they had to be run {\sl outside}
\co{latex} and {\sl before} \co{latex}. With dednat6 this changed;
dednat6 can still be run as a preprocessor, but the recommended way to
run it on, say, \co{foo.tex}, is to put a line like
%
\begin{verbatim}
\directlua{dofile "dednat6load.lua"}
\end{verbatim}
%
somewhere near the beginning of \co{foo.tex}, add some calls to \co{\\pu} at
some points \Dash as we will explain soon \Dash and compile \co{foo.tex} with
\co{lualatex} instead of \co{latex}, to make \co{foo.tex} be processed ``in
parallel'' by \TeX{} and by Lua. That ``in parallel'' is a
simplification, though; consider this example:

\begin{verbatim}
%:
%:  a  b
%:  ----
%:   c
%:
%:   ^my-tree
%:
$$\pu\ded{my-tree}$$
%:
%:  d  e  f
%:  -------
%:     g
%:
%:     ^my-tree
%:
$$\pu\ded{my-tree}$$
\end{verbatim}

Suppose that this fragment starts at line 20. (As mentioned above, we
are omitting the
header and footer \Dash e.g., \co{\\begin\{document\}} and
\co{\\directlua \{dofile "dednat6load.lua"\}}.)

We have a \co{\%:}-block from lines 20--26, a call to \co{\\pu} at
line 27, another \co{\%:}-block from lines 28-34, and another call to
\co{\\pu} at line 35.

The output of the first \co{\%:}-block above is a
\co{\\defded\{my-tree\}}, and the output of the second \co{\%:}-block
above is a {\sl different} \co{\\defded\{my-tree\}}. 

`\co{\\pu}' means ``process until'' \Dash or, more precisely, {\sl make
  dednat6 process everything until this point that it hasn't processed
  yet}. The first \co{\\pu} processes the lines 1--26 of \co{foo.tex}, and
``outputs'' \Dash i.e., sends to \TeX \Dash the first
\co{\\defded\{my-tree\}}; the second \co{\\pu} processes the lines
28--34 of \co{foo.tex}, and ``outputs'' the second
\co{\\defded\{my-tree\}}. Thus, it is not technically true that \TeX\ and dednat6
process \co{foo.tex} in parallel; dednat6 goes later, and each \co{\\pu} is
a synchronization point.



%  _   _                _                       _   _     _            _        
% | | | | ___  __ _  __| |___    __ _ _ __   __| | | |__ | | ___   ___| | _____ 
% | |_| |/ _ \/ _` |/ _` / __|  / _` | '_ \ / _` | | '_ \| |/ _ \ / __| |/ / __|
% |  _  |  __/ (_| | (_| \__ \ | (_| | | | | (_| | | |_) | | (_) | (__|   <\__ \
% |_| |_|\___|\__,_|\__,_|___/  \__,_|_| |_|\__,_| |_.__/|_|\___/ \___|_|\_\___/
%                                                                               
% «heads-and-blocks» (to ".heads-and-blocks")
\subsection{Heads and blocks}

In order to understand how this idea \Dash ``semi-pre\-pro\-cessors''
\Dash is implemented in dednat6 we need some terminology.

The initial {\sl set of heads} is \co{\{"\%:", "\%L", "\%D"\}}. It may
be extended with other heads, but we may only add heads that
start with `\co{\%}'.

A {\sl block} is a set of contiguous lines in the current \co{.tex} file.
This code
%
\begin{verbatim}
Block {i=42, j=99}
\end{verbatim}
%
creates and returns a block that starts on line 42 and ends on line
99. The Lua function \co{Block} receives a table, changes its
metatable to make it a ``block object'', and returns the modified
table.

A {\sl head block} is a (maximal) set of contiguous lines all with
same head. Head blocks are implemented as blocks with an extra field
\co{head}. For example:
%
\begin{verbatim}
Block {i=20, j=26, head="%:"}
\end{verbatim}

A block is {\sl bad} when it contains a part of a head block but not
the whole of it. We avoid dealing with bad blocks \Dash dednat6 never
creates a block object that is ``bad''.

Each head has a {\sl processor}. {\sl Executing} a head block means
running it through the processor associated with its head. Executing an
arbitrary (non-bad) block means executing each head block in it, one at
a time, in order. Note: the code for executing non-bad arbitrary
blocks was a bit tricky to implement, as executing a `\co{\%L}'-block
may change the set of heads and the processors associated to heads.

A {\sl texfile block} is a block that refers to the whole of the
current \co{.tex} file, and that has an extra field \co{nline} that points
to the first line that dednat6 hasn't processed yet. If \co{foo.tex} has
234 lines then the texfile block for \co{foo.tex} starts as:
%
\begin{verbatim}
Block {i=1, j=234, nline=1}
\end{verbatim}

We saw in sections \ref{dednat.icn} and \ref{2D-code} that the
``output'' of a \co{\%:}-block is a series of `\co{\\defded}'s and the
``output'' of a \co{\%D}-block is a series of `\co{\\defdiags}'s. We
can generalize this. For example, the ``output'' of
%
\begin{verbatim}
%L output [[\def\Foo{FOO}]]
%L output [[\def\Bar{BAR}]]
\end{verbatim}
%
is
%
\begin{verbatim}
\def\Foo{FOO}
\def\Bar{BAR}
\end{verbatim}

The {\sl output} of a head block is the concatenation of the strings
sent to \co{output()} when that block is executed. The output of an
arbitrary (non-bad) block is the concatenation of the strings sent to
\co{output()} by its head blocks when the arbitrary block is executed.

A {\sl \co{\\pu}-block} is created by dednat6 when a \co{\\pu} is
executed, pointing to the lines between this \co{\\pu} and the
previous \co{\\pu}. If \co{foo.tex} has a \co{\\pu} at line 27 and another
at line 35 then the first \co{\\pu} creates this block,
%
\begin{verbatim}
Block {i=1, j=26}
\end{verbatim}
%
and the second \co{\\pu} creates this:
%
\begin{verbatim}
Block {i=28, j=34}
\end{verbatim}

As `\co{\\pu}'s only happen in non-comment lines, \co{\\pu}-blocks are
never bad.



%  ___                 _                           _    __                
% |_ _|_ __ ___  _ __ | | ___ _ __ ___   ___ _ __ | |_  \ \   _ __  _   _ 
%  | || '_ ` _ \| '_ \| |/ _ \ '_ ` _ \ / _ \ '_ \| __|  \ \ | '_ \| | | |
%  | || | | | | | |_) | |  __/ | | | | |  __/ | | | |_    \ \| |_) | |_| |
% |___|_| |_| |_| .__/|_|\___|_| |_| |_|\___|_| |_|\__|    \_\ .__/ \__,_|
%               |_|                                          |_|          
%
\subsection{The implementation of \co{\\pu}}
% «implementation-of-pu» (to ".implementation-of-pu")

The macro \co{\\pu} is defined as 
%
\begin{verbatim}
\def\pu{\directlua{
    processuntil(tex.inputlineno)
  }}
\end{verbatim}
%
in \LaTeX, and \co{processuntil()} is this (in Lua):
%
\begin{verbatim}
processuntil = function (puline)
    local publock =
      Block {i=tf.nline, j=puline-1}
    publock:process()
    tf.nline = puline + 1
  end
\end{verbatim}

Here's a high-level explanation. When dednat6 is loaded and
initialized it creates a texfile block for the current \co{.tex} file \Dash
with \co{nline=1} \Dash and stores it in the global variable \co{tf}.
The macro \co{\\pu} creates a \co{\\pu}-block that starts at line
\co{tf.nline} and ends at line \co{tex.inputlineno - 1}, executes it,
and advances \co{tf.nline} \Dash i.e., sets it to
\co{tex.inputlineno + 1}.

% \co{tf.nline}

The code above looks simple because the line \co{publock:process()}
does all the hard work.




%   ____                _   _               _                    _     
%  / ___|_ __ ___  __ _| |_(_)_ __   __ _  | |__   ___  __ _  __| |___ 
% | |   | '__/ _ \/ _` | __| | '_ \ / _` | | '_ \ / _ \/ _` |/ _` / __|
% | |___| | |  __/ (_| | |_| | | | | (_| | | | | |  __/ (_| | (_| \__ \
%  \____|_|  \___|\__,_|\__|_|_| |_|\__, | |_| |_|\___|\__,_|\__,_|___/
%                                   |___/                              
%
% «creating-new-heads» (to ".creating-new-heads")
\section{Creating new heads}

New heads can be created with \co{registerhead}, and they are
recognized immediately. For example, this

\begin{verbatim}
%L eval = function (str)
%L     return assert(loadstring(str))()
%L   end
%L expr = function (str)
%L     return eval("return "..str)
%L   end
%L
%L registerhead "%A" {
%L   name   = "eval-angle-brackets",
%L   action = function ()
%L       local i,j,str = tf:getblockstr()
%L       str = str:gsub("<(.-)>", expr)
%L       output(str)
%L     end,
%L }
%A $2+3 = <2+3>$
\pu
\end{verbatim}
%
\setbox0=\hbox{\pu}   % Discard the output of the real \pu
%
produces ``$2+3=5$''; that looks trivial, but it is easy to write
bigger examples of `\co{\%A}'-blocks with \co{pict2e} code in them, in
which the Lua expressions in `\co{<...>}'s generate
`\co{\\polyline}'s and `\co{\\puts}'s whose coordinates are all
calculated by Lua.





% (find-fline "c:/Users/Vermelhinho/Downloads/2018tugboat.tex" "registerhead")
% (find-dn6 "block.lua")
% (find-dn6 "heads6.lua")







%  ____  _____ ____  _     
% |  _ \| ____|  _ \| |    
% | |_) |  _| | |_) | |    
% |  _ <| |___|  __/| |___ 
% |_| \_\_____|_|   |_____|
%                          
% «REPL» (to ".REPL")
\section{A read-eval-print-loop (\acro{REPL})}

% (find-dn6 "luarepl.lua")
% (tugp 22 "repls-2")
% (tug     "repls-2")

Dednat6 uses only one function from the Lua\TeX{} libraries \Dash
\co{tex.print} \Dash and two variables, \co{status.}\allowbreak\co{filename} and
\co{tex.inputlineno}, but it includes a nice way to play with the
other functions and variables in the libraries.

Dednat6 includes a copy of \co{lua-repl} (by Rob Hoelz,
\url{github.com/hoelzro/lua-repl}), and we can
invoke it by running \co{luarepl()}. If we put this in our \co{foo.tex},
{\hfuzz=1.5pt\par}

\begin{verbatim}
\setbox0=\hbox{abc}
\directlua{luarepl()}
\end{verbatim}
%
then running \co{lualatex foo.tex} will print lots of stuff, and then
the prompt `\co{>>>}' of the \co{lua-repl} inside dednat6; if we send
these commands to the \acro{REPL},

\begin{verbatim}[\footnotesize]
print(tex.box[0])
print(tex.box[0].id,        node.id("hlist"))
print(tex.box[0].list)
print(tex.box[0].list.id,   node.id("glyph"))
print(tex.box[0].list.char, string.byte("a"))
print(tex.box[0].list.next)
print(tex.box[0].list.next.char,
                            string.byte("b"))
\end{verbatim}
%
\newpage\noindent
we get this in the terminal:

\begin{verbatim}[\footnotesize]
>>> print(tex.box[0])
<node    nil <  35981 >    nil : hlist 2>
>>> print(tex.box[0].id,        node.id("hlist"))
0	0
>>> print(tex.box[0].list)
<node    nil <   6107 >   6114 : glyph 256>
>>> print(tex.box[0].list.id,   node.id("glyph"))
29	29
>>> print(tex.box[0].list.char, string.byte("a"))
97	97
>>> print(tex.box[0].list.next)
<node   6107 <   6114 >  32849 : glyph 256>
>>> print(tex.box[0].list.next.char,
>>>>                             string.byte("b"))
98	98
>>>
\end{verbatim}

The best way to use \co{luarepl()} \Dash in my not so humble opinion \Dash
is from Emacs, with the \co{eev} library. The tutorial of eev at

\begin{verbatim}
  http://angg.twu.net/eev-intros/
    find-eev-quick-intro.html
\end{verbatim}
%
explains, in the section ``Controlling shell-like programs'', how we
can edit the commands to be sent to \co{lualatex} in a buffer, called
the ``notes buffer'', and send them line by line to another buffer
that runs \co{lualatex foo.tex} in a shell \Dash the ``target buffer'';
each time that we type the F8 key Emacs sends the current line to the
program running in the target buffer, {\sl as if the user had typed
  it}.





%%%%%%%
%%%%%%% Test the REPL
%%%%%%% See section 6 ("Controlling shell-like programs")
%%%%%%% of: http://angg.twu.net/eev-intros/find-eev-quick-intro.html
%%%%%%% Uncomment the "\directlua" line, run the eepitch block,
%%%%%%% then put the "%" back. 
%%%%%%%

\setbox0=\hbox{abc}
% \directlua{luarepl()}

\def\IGNORETHIS{
 (eepitch-shell)
 (eepitch-kill)
 (eepitch-shell)
lualatex 2018tugboat.tex
print(tex.box[0])
print(tex.box[0].id,        node.id("hlist"))
print(tex.box[0].list)
print(tex.box[0].list.id,   node.id("glyph"))
print(tex.box[0].list.char, string.byte("a"))
print(tex.box[0].list.next)
print(tex.box[0].list.next.char,
                            string.byte("b"))
}







%     _             _ _       _     _ _ _ _         
%    / \__   ____ _(_) | __ _| |__ (_) (_) |_ _   _ 
%   / _ \ \ / / _` | | |/ _` | '_ \| | | | __| | | |
%  / ___ \ V / (_| | | | (_| | |_) | | | | |_| |_| |
% /_/   \_\_/ \__,_|_|_|\__,_|_.__/|_|_|_|\__|\__, |
%                                             |___/ 
%
% «availability» (to ".availability")
\section{Availability}

Dednat6 is not in \CTAN\ yet (as of October, 2018). Until it gets there
you can download it from:

\begin{verbatim}
  http://angg.twu.net/dednat6.html
\end{verbatim}



%  ____       __                                   
% |  _ \ ___ / _| ___ _ __ ___ _ __   ___ ___  ___ 
% | |_) / _ \ |_ / _ \ '__/ _ \ '_ \ / __/ _ \/ __|
% |  _ <  __/  _|  __/ | |  __/ | | | (_|  __/\__ \
% |_| \_\___|_|  \___|_|  \___|_| |_|\___\___||___/
%                                                  
% «references» (to ".references")
\section*{References}

% http://angg.twu.net/miniforth/miniforth-article.pdf
% (find-LATEX "catsem.bib" "bib-Bootstrapping")
% (find-LATEX "catsem.bib" "bib-LuaGems")

{\frenchspacing
[1] E.\ Ochs: {\sl Bootstrapping a Forth in 40 Lines of Lua Code}.
Chapter 6 (pp.\ 57--70) of {\sl Lua Programming Gems}, L.H. de
Figueiredo, W.\ Celes, and R.\ Ierusa\-limschy, eds. {\tt lua.org/gems},
2008. Available from \url{http://angg.twu.net/miniforth-article.html}.

}

% (find-angg "dednat/")
% (find-angg "dednat4/")




\advance\signaturewidth by 8pt
\makesignature
\end{document}

% Local Variables:
% coding: utf-8-unix
% End:
