% (find-angg "dednat6/2018dednat6-extras.tex")
% (defun c () (interactive) (find-dednat6sh "lualatex -record 2018dednat6-extras.tex"))
% (defun d () (interactive) (find-pdf-page "~/dednat6/2018dednat6-extras.pdf"))
% (defun e () (interactive) (find-dednat6 "2018dednat6-extras.tex"))
% (defun l () (interactive) (find-dednat6 "myverbatim.lua"))
% (defun u () (interactive) (find-latex-upload-links "2018dednat6-extras"))
% (find-pdf-page   "~/dednat6/2018dednat6-extras.pdf")
% (find-sh0 "cp -v  ~/dednat6/2018dednat6-extras.pdf /tmp/")
% (find-sh0 "cp -v  ~/dednat6/2018dednat6-extras.pdf /tmp/pen/")
%   file:///home/edrx/dednat6/2018dednat6-extras.pdf
%                 file:///tmp/2018dednat6-extras.pdf
%             file:///tmp/pen/2018dednat6-extras.pdf
% http://angg.twu.net/dednat6/2018dednat6-extras.pdf

% «.verbatim»		(to "verbatim")
%
% «.title»		(to "title")
% «.what-is-this»	(to "what-is-this")
% «.downloading»	(to "downloading")
%   «.dednat6.zip-test»	  (to "dednat6.zip-test")
% «.preamble»		(to "preamble")
% «.tree-bars»		(to "tree-bars")
% «.tree-hacks»		(to "tree-hacks")
% «.abbrevs»		(to "abbrevs")
% «.renaming»		(to "renaming")
% «.arrow-modifiers»	(to "arrow-modifiers")
% «.strange-modules»	(to "strange-modules")
% «.barr»		(to "barr")
% «.a-few-samples»	(to "a-few-samples")
% «.repl»		(to "repl")
% «.other-back-ends»	(to "other-back-ends")

\documentclass[oneside]{article}
\usepackage[colorlinks]{hyperref} % (find-es "tex" "hyperref")
\usepackage{amsmath}
\usepackage{amsfonts}
\usepackage{amssymb}
\usepackage{pict2e}
\usepackage{color}                % (find-dednat6 "edrx15.sty" "colors")
\usepackage{colorweb}             % (find-es "tex" "colorweb")
\usepackage{graphicx}    % <- temporary?
%\usepackage{tikz}
%
% (find-dn6 "preamble6.lua" "preamble0")
\usepackage{proof}   % For derivation trees ("%:" lines)
\input diagxy        % For 2D diagrams ("%D" lines)
\xyoption{curve}     % For the ".curve=" feature in 2D diagrams
\catcode`\^^J=10                      % (find-es "luatex" "spurious-omega")
\directlua{dofile "dednat6load.lua"}  % (find-dednat6 "dednat6load.lua")
\def\expr#1{\directlua{output(tostring(#1))}}
\def\eval#1{\directlua{#1}}
%
%\usepackage{edrx15}               % (find-angg "dednat6/edrx15.sty")
%\input edrx15defs.tex         % (find-LATEX "edrx15defs.tex")
%\input edrxaccents.tex            % (find-angg "dednat6/edrxaccents.tex")
%\input edrxchars.tex              % (find-dednat6 "edrxchars.tex")
%\input edrxheadfoot.tex           % (find-dn4ex "edrxheadfoot.tex")
%\input edrxgac2.tex               % (find-dednat6 "edrxgac2.tex")
\def\bsk{\bigskip}
\def\msk{\medskip}
\def\ssk{\smallskip}
\def\orange#1{{\color{orange}#1}}
\def\yellow#1{{\color{yellow}#1}}
\def\smashedvrule#1#2#3{\vrule width#1 height#2 depth#3 \kern-#1}
\def\bicolorhbox#1{{%
  \setbox0\hbox{#1}%
  \yellow{\smashedvrule{\wd0}{\ht0}{0pt}}%
  \orange{\smashedvrule{\wd0}{0pt}{\dp0}}%
  \box0%
  }}
\def\bhbox{\bicolorhbox}
\def\cded #1{\begin{matrix}\ded {#1}\end{matrix}}
%
\begin{document}

\catcode`\^^J=10
%\directlua{dofile "edrxtikz.lua"} % (find-dednat6 "edrxtikz.lua")
%\directlua{dofile "edrxpict.lua"} % (find-dednat6 "edrxpict.lua")
% %L V.__tostring = function (v) return format("(%.3f,%.3f)", v[1], v[2]) end


\def\ASCII{ASCII}

% \co: a low-level way to typeset code; a poor man's "\verb"
\def\co#1{{%
  \def\%{\char37}%
  \def\\{\char92}%
  \def\^{\char94}%
  \def\~{\char126}%
  \tt#1%
  }}
\def\qco#1{`\co{#1}'}
\def\qqco#1{``\co{#1}''}


% «verbatim» (to ".verbatim")
% (find-dednat6 "2018tug-dednat6.tex" "verbatim")
\directlua{dofile "myverbatim.lua"} % (find-dednat6 "myverbatim.lua")
\def\myhbox#1#2#3{\setbox0=\hbox{#3}\ht0=#1\dp0=#2\box0}
\def\verbahbox#1{\myhbox{7pt}{2pt}{{\tt#1}}}
\def\bgbox#1{\bgcolorhbox{YellowOrangeLight}{#1}}
\def\bgcolorhbox#1#2{{%
  \setbox0\hbox{#2}%
  \setbox0\vbox{\vskip\fboxsep\box0\vskip\fboxsep}%
  \setbox0\hbox{\kern\fboxsep\box0\kern\fboxsep}%
  {\color{#1}{\smashedvrule{\wd0}{\ht0}{\dp0}}}%
  \box0%
  }}
%
\long\def\ColorRed   #1{{\color{Red}#1}}
\long\def\ColorViolet#1{{\color{MagentaVioletLight}#1}}
\long\def\ColorGreen #1{{\color{SpringDarkHard}#1}}
\long\def\ColorGreen #1{{\color{SpringGreenDark}#1}}
\long\def\ColorGray  #1{{\color{GrayLight}#1}}
%
\catcode`𝐚=13 \def𝐚{\color{SpringGreenDark}}
\catcode`𝐛=13 \def𝐛{\color{black}}
\catcode`𝐠=13 \def𝐠{\color{SpringGreenDark}}
\catcode`𝐫=13 \def𝐫{\color{Red}}
\catcode`𝐧=13 \def𝐧{\color{blue}}
\catcode`𝐮=13 \def𝐮{\color{blue}}
\catcode`𝐜=13 \def𝐜{\color{pink}}
\catcode`𝐰=13 \def𝐰{\color{YellowOrangeLight}}

% (find-dednat6 "edrxchars.tex")
\catcode`Ω=13 \defΩ{\Omega}
\catcode`Γ=13 \defΓ{\Gamma}
\catcode`⊢=13 \def⊢{\vdash}
\catcode`∨=13 \def∨{\lor}
\catcode`→=13 \def→{\to}


\def\myvcenter#1{\begin{matrix}#1\end{matrix}}
\catcode`\^^O=13 \def{{\color{red}*}}





%  _____ _ _   _
% |_   _(_) |_| | ___
%   | | | | __| |/ _ \
%   | | | | |_| |  __/
%   |_| |_|\__|_|\___|
%
% «title» (to ".title")
% (find-es "tex" "title")
% (dnep 1)

\title{Dednat6: Some Extra Features}

\author{Eduardo Ochs}

\maketitle

% $$\bfig
%   \morphism(2400,0)|l|/{@{>}@`{c,(3000,0),(2700,2800),p}}/<-1200,2800>[A`B;ab]
%   \morphism(0,0)|l|/{@{>}@`{c,(3000,0),(2700,2800),p}}/<400,200>[C`D;cd]
%   \morphism(0,0)|l|/{@{>}@`{c,(1000,0),(600,200),p}}/<400,200>[C`D;cd]
%   \efig
% $$
% 
% \end{document}

% __        ___           _     _       _   _     _
% \ \      / / |__   __ _| |_  (_)___  | |_| |__ (_)___
%  \ \ /\ / /| '_ \ / _` | __| | / __| | __| '_ \| / __|
%   \ V  V / | | | | (_| | |_  | \__ \ | |_| | | | \__ \
%    \_/\_/  |_| |_|\__,_|\__| |_|___/  \__|_| |_|_|___/
%
% «what-is-this» (to ".what-is-this")
% \section{What this document is}

At this moment the documentation of Dednat6 consists of:

\begin{itemize}

\item An article about Dednat6 for TUGBoat, called ``Dednat6: An
  extensible (semi-)preprocessor for Lua\LaTeX\ that understands
  diagrams in \ASCII\ art'', that explains all the main concepts and
  how they are implemented. Link:

  \url{http://angg.twu.net/LATEX/2018tugboat-rev1.pdf}

\item The slides for a presentation with the same title (on the
  TUG2018 conference). Link:

  \url{http://angg.twu.net/LATEX/2018tug-dednat6.pdf}

\item {\sl This document}, that complements the article and the slides
  --- with installation instructions, plus lots of details and
  technicalities. Link:

  \url{http://angg.twu.net/LATEX/2018dednat6-extras.pdf} \\

\end{itemize}

See also:

  % (find-TH "math-b" "tug-2018")
  \url{http://angg.twu.net/dednat6.html}

  \url{http://angg.twu.net/math-b.html\#tug-2018}



\section{Hey!}

The slides for the TUG talk end with this call for feedback:

\msk

I've stopped trying to document Dednat6 because

1) I don't have a mental image of who I am writing for,

2) I get {\sl far too little feedback},

3) all of the feedback that I got came from people who felt that I was
not writing for them --- my approach, tone and choice of
pre-requisites were all wrong.

If you would like to try Dednat6, get in touch, {\bf let's chat} ---
{\sl please}!

{\sl 
Maybe I can typeset in 20 minutes a diagram that took you

a day, maybe I can implement an extension that you need...
}

\msk

This still holds! I'm \co{eduardoochs@gmail.com}, get in touch!



Index of sections:

{\makeatletter
\renewcommand*\l@section{\@dottedtocline{1}{1.5em}{2.3em}}
\@starttoc{toc}
}




%  ____                      _                 _ _
% |  _ \  _____      ___ __ | | ___   __ _  __| (_)_ __   __ _
% | | | |/ _ \ \ /\ / / '_ \| |/ _ \ / _` |/ _` | | '_ \ / _` |
% | |_| | (_) \ V  V /| | | | | (_) | (_| | (_| | | | | | (_| |
% |____/ \___/ \_/\_/ |_| |_|_|\___/ \__,_|\__,_|_|_| |_|\__, |
%                                                        |___/
% «downloading» (to ".downloading")
\section{Downloading}

You can download the current version of dednat6 from
\url{http://angg.twu.net/dednat6.html}, or from here:
\url{http://angg.twu.net/dednat6.zip}. The .zip file contains
documentation in PDF and source form. To download it, delete the PDFs
and recompile them, do this:

% «dednat6.zip-test» (to ".dednat6.zip-test")
% (find-es "dednat" "dednat6.zip-test")

%V rm -Rfv /tmp/dn6-test/
%V mkdir   /tmp/dn6-test/
%V cd      /tmp/dn6-test/
%V wget http://angg.twu.net/dednat6.zip
%V unzip dednat6.zip
%V rm -v *.pdf
%V lualatex 2018tugboat-rev1.tex
%V lualatex 2018tugboat-rev1.tex
%V lualatex 2018tug-dednat6.tex
%V lualatex 2018tug-dednat6.tex
%V lualatex 2018dednat6-extras.tex
%V lualatex 2018dednat6-extras.tex
%V lualatex 2018dednat6-minimal.tex
%V lualatex 2018dednat6-no-lua.tex
%V mkdir no-lua/
%V cd    no-lua/
%V cp -v ../2018dednat6-no-lua.tex ../2018dednat6-no-lua.dnt .
%V pdflatex 2018dednat6-no-lua.tex
%
%L verbdef "foo"
%
$\pu\foo$

Dednat6 itself consists of just the contents of the \co{dednat6/}
directory plus the file \co{dednat6load.lua}. You can run the code
below to check that that's what is needed to compile
\co{2018dednat6-minimal.tex}:

%V rm -Rfv /tmp/dn6-test-min/
%V mkdir   /tmp/dn6-test-min/
%V cd      /tmp/dn6-test-min/
%V wget http://angg.twu.net/dednat6.zip
%V unzip dednat6.zip "dednat6/**" dednat6load.lua 2018dednat6-minimal.tex
%V lualatex 2018dednat6-minimal.tex
%
%L verbdef "foo"
%
$\pu\footnotesize\foo$





%V lualatex 2018dednat6-minimal.tex


%  ____                           _     _
% |  _ \ _ __ ___  __ _ _ __ ___ | |__ | | ___
% | |_) | '__/ _ \/ _` | '_ ` _ \| '_ \| |/ _ \
% |  __/| | |  __/ (_| | | | | | | |_) | |  __/
% |_|   |_|  \___|\__,_|_| |_| |_|_.__/|_|\___|
%
% «preamble» (to ".preamble")

\section{The preamble}

It {\sl should} be possible to load Dednat6 from a .tex file with a
single command, like this:

\msk

%V \documentclass{article}
%V   \directlua{dofile "dednat6load.lua"}
%V \begin{document}
%V (...)
%V \end{document}
%
%L verbdef "foo"
%
$\pu\foo$

\msk

\noindent but due to some quirks this is not possible at this moment,
and we have to do this (the indented lines):

\msk

%V \documentclass{article}
%V   \usepackage{proof}
%V   \input diagxy
%V   \xyoption{curve}
%V \begin{document}
%V   \catcode`\^^J=10
%V   \directlua{dofile "dednat6load.lua"}
%V (...)
%V \end{document}
%
%L verbdef "foo"
%
$\pu\foo$

\msk

% (find-es "luatex" "spurious-omega")

The `\co{\\catcode}' is needed to make the newlines in the \TeX{} code
generated by dednat6 be interpreted as newlines and not as `$Ω$'s (the
``spurious omega problem''); `\co{\\usepackage\{proof\}}' loads a
package for typesetting proof trees, `\co{\\input diagxy}' loads \Xy
Pic and the \co{diagxy} extension (sec.\ref{diagxy}), and
\co{\\xyoption{curve}} loads an extension for \co{diagxy} that allows
drawing curved arrows.

% (find-dn6 "preamble6.lua" "preamble1")

One of the things that \co{\\directlua\{dofile "dednat6load.lua"\}}
does is that it runs this from Lua:
%
$$\co{output(preamble1)}$$
%
\co{preamble1} is a chunk of \TeX{} code defined in the file
\co{dednat6/preamble.lua}, that contains material like this:

\msk

%V \def\defdiag#1#2{\expandafter\def\csname diag-#1\endcsname{\bfig#2\efig}}
%V \def\ifdiagundefined#1{\expandafter\ifx\csname diag-#1\endcsname\relax}
%V \def\diag#1{\ifdiagundefined{#1}
%V     \errmessage{UNDEFINED DIAGRAM: #1}
%V   \else
%V     \csname diag-#1\endcsname
%V   \fi
%V }
%V %
%V \def\expr#1{\directlua{output(tostring(#1))}}
%V \def\eval#1{\directlua{#1}}
%V \def\pu{\directlua{pu()}}
%
%L verbdef "foo"
%
$\pu\footnotesize\foo$

The file \co{dednat6/preamble.lua} also defines a `\co{preamble0}'
with the \qco{\\usepackage}s and \qco{\\input}s that would have to be
run before the \qco{\\begin\{document\}}, and it explains in comments
why this currently doesn't work. See the source.



%  _   _         _                
% | \ | | ___   | |   _   _  __ _ 
% |  \| |/ _ \  | |  | | | |/ _` |
% | |\  | (_) | | |__| |_| | (_| |
% |_| \_|\___/  |_____\__,_|\__,_|
%                                 
% «no-lua» (to ".no-lua")

\section{Producing a .tex/.dnt pair that doesn't need Lua\LaTeX}

Here is a trick for dealing with situations in which we need to
generate code that compiles with just \co{pdflatex}, without
\co{lualatex} --- for example, when we need to produce \LaTeX{} code
acceptable by Arxiv. The file \co{2018dednat6-no-lua.tex} in the .zip
package is an example of that trick. To prepare a version of it that
only needs \co{pdflatex}, do this:

\msk

%V rm -Rfv /tmp/dn6-test-no-lua/
%V mkdir   /tmp/dn6-test-no-lua/
%V cd      /tmp/dn6-test-no-lua/
%V wget http://angg.twu.net/dednat6.zip
%V unzip dednat6.zip "dednat6/**" dednat6load.lua 2018dednat6-no-lua.tex
%V lualatex 2018dednat6-no-lua.tex
%V mkdir no-lua/
%V cd    no-lua/
%V cp -v ../2018dednat6-no-lua.tex ../2018dednat6-no-lua.dnt .
%V pdflatex 2018dednat6-no-lua.tex
%V xpdf     2018dednat6-no-lua.pdf
%
%L verbdef "foo"
%
$\pu\footnotesize\foo$

\msk

The line \qqco{lualatex 2018dednat6-no-lua.tex} generates a \co{.dnt}
file; the commands after that create a directory with just the
\co{.tex} and the \co{.dnt}, and compiles the \co{.tex} with
\co{pdflatex}.

A \co{.tex} file that supports being compiled in this was has this
structure:

\msk

%V \documentclass[oneside]{book}
%V \usepackage{ifluatex}
%V \usepackage{proof}
%V \input diagxy
%V \xyoption{curve}
%V \begin{document}
%V 
%V \ifluatex
%V   \catcode`\^^J=10
%V   \directlua{dofile "dednat6load.lua"}
%V \else
%V   \input\jobname.dnt
%V   \def\pu{}
%V \fi
%V 
%V (...)
%V 
%V %L write_dnt_file()
%V \pu
%V 
%V \end{document}
%
%L verbdef "foo"
%
{\pu\foo}

\msk

Note the \qqco{\\usepackage\{ifluatex\}}, the \co{\\ifluatex} /
\co{\\else} / \co{\\fi} block, and the \qqco{\%L write\_dnt\_file()}
followed by a \co{\\pu}.






%  _____                _                    
% |_   _| __ ___  ___  | |__   __ _ _ __ ___ 
%   | || '__/ _ \/ _ \ | '_ \ / _` | '__/ __|
%   | || | |  __/  __/ | |_) | (_| | |  \__ \
%   |_||_|  \___|\___| |_.__/ \__,_|_|  |___/
%                                            
% «tree-bars» (to ".tree-bars")

\section{Other inference bars}

% (find-es "tex" "proof")
% (find-LATEXfile "2010hyps.tex" "\\DeduceSym")

All the examples of deduction trees in the TUGBoat article use
`\co{-}'s for the inference bars in the ASCII art representation. If
we use `\co{=}'s instead of `\co{-}'s we get double bars, and if we
use `\co{:}'s we get a line of vertical dots instead of a bar:
%:
%:                  Γ    Γ  [P]^1  Γ\;\;[Q]^1
%:                  :    ::::::::  ::::::::::
%:  Γ,P⊢R  Γ,Q⊢R    P∨Q    R           R
%:  ============    --------------------1
%:    Γ,P∨Q⊢R               R
%:
%:    ^or-1                 ^or-2
%:
$$\pu \ded{or-1} \quad\Longrightarrow\quad \ded{or-2}$$

You can change the number of vertical dots by redefining the macro
`\co{\\DeduceSym}'. For example:

\msk

%V \makeatletter
%V % Original with 4 dots (from proof.sty):
%V % \def\DeduceSym{\vtop{\baselineskip4\p@ \lineskiplimit\z@
%V %     \vbox{\hbox{.}\hbox{.}\hbox{.}}\hbox{.}}}
%V % New, with 3 dots:
%V \def\DeduceSym{\vtop{\baselineskip4\p@ \lineskiplimit\z@
%V     \vbox{\hbox{.}\hbox{.}}\hbox{.}}}
%V \makeatother
%
%L verbdef "foo"
%
$\pu\foo$




%  _____                _                _        
% |_   _| __ ___  ___  | |__   __ _  ___| | _____ 
%   | || '__/ _ \/ _ \ | '_ \ / _` |/ __| |/ / __|
%   | || | |  __/  __/ | | | | (_| | (__|   <\__ \
%   |_||_|  \___|\___| |_| |_|\__,_|\___|_|\_\___/
%                                                 
% «tree-hacks» (to ".tree-hacks")
\section{Changing how tree nodes are \LaTeX'ed}

The trees below were produced by changing {\sl temporarily} the
function that converts each tree node from ascii to \TeX{} code. TO
DO: explain this; for the moment look at the source code of this
section.

% The functions that generate \TeX{} code for deduction trees are in
% the file \co{dednat6/treetex.lua}. Dednat6 uses Makoto Tatsuta's
% \co{proof.sty} by default, and the functions that generate code
% using \co{proof.sty}'s macros are \co{TeX\_subtree\_tatsuta} and
% \co{TeX\_deftree\_tatsuta}. They call \co{unabbrev\_tatsuta} to
% generate \TeX{} code for each node,

% (find-dn6 "treetex.lua" "unabbrev_tatsuta")

%L unabbrev_syntree = function (str)
%L     if str:match"^<.*>$"
%L     then return "\\langle\\text{"..str:sub(2, -2).."}\\rangle"
%L     else return "\\co{"..str:gsub("[\\]", "\\%1").."}"
%L     end
%L   end
%L unabbrev_tatsuta = unabbrev_syntree
%L -- To deactivate the hack do:
%L -- unabbrev_tatsuta = unabbrev


% \POS(1500,1000)*+
% !!<0ex,.75ex>
% {T_{n-1}}\ar@{-}|!{(1000,500);(1500,500)}\hole(1167,334)
% \POS(1167,334)\ar|!{(1000,500);(1000,0)}\hole_<>(.6){h_{n-1}}
%  (500,0)*+!!<0ex,.75ex>{S_n}

%:                                              
%:                                              
%:                                              
%:                                              
%:      1500       1000                         
%:    --------   --------                       
%:  ( <factor> , <factor> )      +    <empty>   
%:  -----------------------   ------- ------    
%:      <vector>              <addop> <size>    
%:      -------               ---------------   
%:      <coord>                 <modifier>      <object1>
%:      -------   --------------------------------------
%:        <pos> * <object>
%:        ----------------
%:   \POS <coord>
%:   ------------
%:    <command>
%:
%:    ^command
%:
%:
%:                0ex      .75ex    
%:              -------   -------   
%:            < <dimen> , <dimen> > 
%:            --------------------- 
%:         !      <vector>          
%:         ---------------          
%:      !       <modifier>  <object2>
%:  ----------  ------------------- 
%:  <modifier>      <object>        
%:  ------------------------
%:       <object1>
%:
%:       ^object1
%:
%:
%:
%:
%:
%:  -----------------
%:       <object2>
%:
%:       ^object2
%:
\pu
$$\ded{command}$$
$$\ded{object1}$$

%-- Revert:
%L unabbrev_tatsuta = unabbrev






%     _    _     _
%    / \  | |__ | |__  _ __ _____   _____
%   / _ \ | '_ \| '_ \| '__/ _ \ \ / / __|
%  / ___ \| |_) | |_) | | |  __/\ V /\__ \
% /_/   \_\_.__/|_.__/|_|  \___| \_/ |___/
%
% «abbrevs» (to ".abbrevs")
% (find-dn6file "dednat6.lua" "abbrevs")
% (find-dn6 "abbrevs.lua")

\section{Abbrevs}

The first Dednats did not support UTF-8, and the way to write a tree
node that would display as `$a→b$' was to write it as `\co{a->b}'
after running \co{addabbrevs("->", "\\to ")}. The module
\co{abbrevs.lua} implements this, and \co{unabbrev(str)} parses
\co{str} from left to right, at each point looking for the longest
string starting at that point that is an abbrev and replacing it by
its expansion, or leaving that character untouched if it doesn't have
an expansion. Here is an example:

%V %L addabbrevs("->", "\\to ")
%V %:
%V %:  [a]^1  a->b
%V %:  -----------
%V %:       b       b->c
%V %:       ------------
%V %:            c
%V %:          ----1
%V %:          a->c
%V %:
%V %:          ^a->c
%V %:
%V $$\pu \ded{a->c}$$
%L
%L verbdef "abbrevcode"

%L addabbrevs("->", "\\to ")
%:
%:  [a]^1  a->b
%:  -----------
%:       b       b->c
%:       ------------
%:            c
%:          ----1
%:          a->c
%:
%:          ^a->c
%:
%$$\pu \ded{a->c}$$

$$\pu
  \myvcenter{\abbrevcode}
  %\abbrevcode
  \qquad
  \to
  \qquad
  \cded{a->c}
$$

% (turp 2)

Abbrevs are also used in 2D diagrams, in a more complex way. Section
2.2 of the TUGBoat article explains how the grid words create a table
\co{nodes}, but it doesn't explain how the fields \co{.tex} and
\co{.TeX} in a node affect how it is displayed. The code below creates
nodes whose {\sl tags} are \co{"A"}, \co{"B"}, \co{"C"}, \co{"D"}, and
then changes the fields \co{.tex} and \co{.TeX} in some of these
nodes. The \TeX{} code for each node is calculated by the function
\co{node\_to\_TeX}, that expects a node (a table) and returns a
string. If \co{node\_to\_TeX} receives a node that has a \co{.TeX}
field then it returns that field unchanged, surrounded by
`\co{\{\}}'s; if it doesn't have a \co{.TeX} field but it has a
\co{.tex} field then it returns the result of running \co{unabbrev} on
that field and surrounding it with `\co{\{\}}'s; otherwise it returns
the result of running \co{unabbrev} on the tag surrounding it with
`\co{\{\}}'s. For example:

%V %D diagram nodes-and-abbrevs
%V %D 2Dx     100 +40
%V %D 2D  100 A -> B
%V %D 2D      |    |
%V %D 2D      v    v
%V %D 2D  +30 C -> D
%V %D 2D
%V %D (( B .tex= (a->b)
%V %D    C              .TeX= (a->b)
%V %D    D .tex= (a->b) .TeX= (a->b)
%V %L print("nodes:"); print(nodes)
%V %L print("A:", node_to_TeX(nodes["A"]))
%V %L print("B:", node_to_TeX(nodes["B"]))
%V %L print("C:", node_to_TeX(nodes["C"]))
%V %L print("D:", node_to_TeX(nodes["D"]))
%V %D    A B -> A C -> B D -> C D ->
%V %D ))
%V %D enddiagram
%L
%L verbdef "abbrevdiag"

%D diagram nodes-and-abbrevs
%D 2Dx     100  +40
%D 2D  100 A -> B
%D 2D      |    |
%D 2D      v    v
%D 2D  +30 C -> D
%D 2D
%D (( B .tex= (a->b)
%D    C              .TeX= (a->b)
%D    D .tex= (a->b) .TeX= (a->b)
%L print("nodes:"); print(nodes)
%L print("A:", node_to_TeX(nodes["A"]))
%L print("B:", node_to_TeX(nodes["B"]))
%L print("C:", node_to_TeX(nodes["C"]))
%L print("D:", node_to_TeX(nodes["D"]))
%D    A B -> A C -> B D -> C D ->
%D ))
%D enddiagram
%D
$$\pu
  \myvcenter{\abbrevdiag}
  \quad
  \to
  \quad
  \diag{nodes-and-abbrevs}
$$

The output of the \co{print()}s is:

{\footnotesize
\begin{verbatim}
nodes:
{   1={"noden"=1, "tag"="A", "x"=100, "y"=100},
    2={"noden"=2, "tag"="B", "x"=140, "y"=100, "tex"="(a->b)"},
    3={"noden"=3, "tag"="C", "x"=100, "y"=130,                 "TeX"="(a->b)"},
    4={"noden"=4, "tag"="D", "x"=140, "y"=130, "tex"="(a->b)", "TeX"="(a->b)"},
  "A"={"noden"=1, "tag"="A", "x"=100, "y"=100},
  "B"={"noden"=2, "tag"="B", "x"=140, "y"=100, "tex"="(a->b)"},
  "C"={"noden"=3, "tag"="C", "x"=100, "y"=130,                 "TeX"="(a->b)"},
  "D"={"noden"=4, "tag"="D", "x"=140, "y"=130, "tex"="(a->b)", "TeX"="(a->b)"}
}
A:      {A}
B:      {(a\to b)}
C:      {(a->b)}
D:      {(a->b)}
\end{verbatim}
}


%  ____                            _
% |  _ \ ___ _ __   __ _ _ __ ___ (_)_ __   __ _
% | |_) / _ \ '_ \ / _` | '_ ` _ \| | '_ \ / _` |
% |  _ <  __/ | | | (_| | | | | | | | | | | (_| |
% |_| \_\___|_| |_|\__,_|_| |_| |_|_|_| |_|\__, |
%                                          |___/
% «renaming» (to ".renaming")
% (find-dn6 "diagforth.lua" "dxyren")
\section{Renaming}

The word \co{ren} in the language for 2D diagrams eats the rest of the
line, splits it at the `\co{==>}', and splits the material before the
`\co{==>}' into a list of {\sl tags}, $A_1, \ldots, A_n$, and the
material after `\co{==>}' into a list of {\sl texs}, $B_1, \ldots,
B_n$; these two lists must have the same length, and then \co{ren}
runs \co{nodes[$A_i$].tex = $B_i$} for each $i$ in $1,\ldots,n$. For
example:

%V %D diagram ren
%V %D 2Dx     100    +30
%V %D 2D  100 A1 <-| A2
%V %D 2D      |       |
%V %D 2D      v       v
%V %D 2D  +30 A3 |-> A4
%V %D 2D
%V %D ren A1 A2 ==> A{\times}B A
%V %D ren A3 A4 ==> C B{\to}C
%V %D
%V %D (( A1 A2 <-| .plabel= a ({\times}B)
%V %D    A1 A3 -> A2 A4 ->
%V %D    A3 A4 |-> .plabel= b (B{\to})
%V %D ))
%V %D enddiagram
%L
%L verbdef "diagren"

%D diagram ren
%D 2Dx     100    +30
%D 2D  100 A1 <-| A2
%D 2D      |       |
%D 2D      v       v
%D 2D  +30 A3 |-> A4
%D 2D
%D ren A1 A2 ==> A{\times}B A
%D ren A3 A4 ==> C B{\to}C
%D
%D (( A1 A2 <-| .plabel= a ({\times}B)
%D    A1 A3 -> A2 A4 ->
%D    A3 A4 |-> .plabel= b (B{\to})
%D ))
%D enddiagram
%D
$$\pu
  \myvcenter{\diagren}
  \quad
  \to
  \quad
  \diag{ren}
$$



%                        _       
%    _ __ ___   ___   __| |_____ 
%   | '_ ` _ \ / _ \ / _` |_____|
%  _| | | | | | (_) | (_| |_____|
% (_)_| |_| |_|\___/ \__,_|      
%                                
% «arrow-modifiers» (to ".arrow-modifiers")
\section{Arrow modifiers}

% (find-es "diagxy" "shape")
% (find-dn6 "diagtex.lua" "arrow_to_TeX")
% (find-dn6 "diagtex.lua" "arrow_to_TeX" "arrow_to_TeX_pshL =")

The language for 2D diagrams in dednat6 has some words for curving and
sliding arrows:
%
%V %D diagram curve-slide
%V %D 2Dx     100 +40
%V %D 2D  100 A   B
%V %D 2D
%V %D 2D  +40 D   C
%V %D 2D
%V %D (( A B ->               .plabel= b \text{one}
%V %D    A B -> .slide=  5pt  .plabel= a \text{two}
%V %D    A B -> .slide= 20pt  .plabel= a \text{three}
%V %D    B C -> .curve= _10pt .plabel= l \text{four}
%V %D    B C -> .curve=  ^5pt .plabel= r \text{five}
%V %D    C D ->
%V %D    C D -> .curve=  _5pt
%V %D    C D -> .curve=  _5pt .slide= -5pt
%V %D           .plabel= a \text{six}
%V %D ))
%V %D enddiagram
%L
%L verbdef "curveslide"
%
%D diagram curve-slide
%D 2Dx     100 +40
%D 2D  100 A   B
%D 2D
%D 2D  +40 D   C
%D 2D
%D (( A B ->               .plabel= b \text{one}
%D    A B -> .slide=  5pt  .plabel= a \text{two}
%D    A B -> .slide= 20pt  .plabel= a \text{three}
%D    B C -> .curve= _10pt .plabel= l \text{four}
%D    B C -> .curve=  ^5pt .plabel= r \text{five}
%D    C D ->
%D    C D -> .curve=  _5pt
%D    C D -> .curve=  _5pt .slide= -5pt
%D           .plabel= a \text{six}
%D ))
%D enddiagram
%D
$$\pu
  \myvcenter{\footnotesize\curveslide}
  \quad
  \to
  \quad
  \diag{curve-slide}
$$

% (find-dn6 "diagforth.lua" "arrows")
% (find-dn6 "diagforth.lua" "arrows" "sl^^")

The words \qco{sl\^\^}, \qco{sl\^}, \qco{sl\_}, and \qco{sl\_\_} are
abbreviations for \qqco{.slide= 5pt}, \qqco{.slide= 2.5pt},
\qqco{.slide= -2.5pt}, \qqco{.slide= -5pt} respectively.

%L forths[".mod="] = function ()
%L     ds:pick(0).modifier = getword() or error()
%L   end

% %D diagram ??
% %D 2Dx     100 +40
% %D 2D  100 A   B
% %D 2D
% %D 2D  +40 D   C
% %D 2D
% %D (( A B -> .plabel= a 1 .mod= @<-5pt>@/_5pt/
% %D
% %D ))
% %D enddiagram
% %D
% $$\pu
%   \diag{??}
% $$





%  ____  _                                          _       _
% / ___|| |_ _ __ _ __   __ _   _ __ ___   ___   __| |_   _| | ___  ___
% \___ \| __| '__| '_ \ / _` | | '_ ` _ \ / _ \ / _` | | | | |/ _ \/ __|
%  ___) | |_| |  | | | | (_| | | | | | | | (_) | (_| | |_| | |  __/\__ \
% |____/ \__|_|  |_| |_|\__, | |_| |_| |_|\___/ \__,_|\__,_|_|\___||___/
%                       |___/
%
% «strange-modules» (to ".strange-modules")
\section{Strange modules}

% (find-dn6 "dednat6.lua")
% (find-dn6 "dednat6.lua" "requires" "picture")
% (find-dn6 "picture.lua")
% (find-dn6 "zhas.lua")
% (find-dn6 "zhas.lua" "MixedPicture")

The code of Dednat6 --- inside the directory \co{dednat6/} --- is made
of several \co{.lua} files that are {\sl all} loaded by
\co{dednat6.lua}; there is no provision yet for loading only the
modules that are used in a given \co{.tex} file. This means that some
modules that are only useful to the author of Dednat6 (Eduardo Ochs,
a.k.a. ``me'') are always loaded.

Most of these extra modules were written to handle the objects
described in my series of papers about ``Planar Heyting Algebras'',
at:

\url{http://angg.twu.net/math-b.html#zhas-for-children-2}

\msk

% (find-LATEX "2017planar-has-defs.tex" "defzha-and-deftcg")
\def\defzha#1#2{\expandafter\def\csname zha-#1\endcsname{#2}}
\def\ifzhaundefined#1{\expandafter\ifx\csname zha-#1\endcsname\relax}
\def\zha#1{\ifzhaundefined{#1}
    \errmessage{UNDEFINED ZHA: #1}
  \else
    \csname zha-#1\endcsname
  \fi
}
\def\deftcg#1#2{\expandafter\def\csname tcg-#1\endcsname{#2}}
\def\iftcgundefined#1{\expandafter\ifx\csname tcg-#1\endcsname\relax}
\def\tcg#1{\iftcgundefined{#1}
    \errmessage{UNDEFINED TCG: #1}
  \else
    \csname tcg-#1\endcsname
  \fi
}

% (find-LATEXfile "2019ebl-five-appls.tex" "A strange J-operator")
%
%L tdims = TCGDims {qrh=5, q=15, crh=12, h=60, v=25, crv=7}   -- with v arrows
%L tspec_PA  = TCGSpec.new("46; 11 22 34 45, 25")
%L tspec_PAQ = TCGSpec.new("46; 11 22 34 45, 25", ".???", "???.?.")
%L tspec_PA :mp  ({zdef="O_A(P)"})  :addlrs():print()            :output()
%L tspec_PAQ:mp  ({zdef="O_A(P),J"}):addlrs():print()            :output()
%L tspec_PA :tcgq({tdef="(P,A)",   meta="1pt p"}, "lr q h v ap") :output()
%L tspec_PAQ:tcgq({tdef="(P,A),Q", meta="1pt p"}, "lr q h v ap") :output()
%L
%L tspec_PAC = TCGSpec.new("46; 11 22 34 45, 25", "?...", "???...")
%L tspec_PAC:mp  ({zdef="closed-op"}) :addlrs():print()            :output()
%L tspec_PAC:tcgq({tdef="closed-op", meta="1pt p"}, "lr q h v ap") :output()
%L 
\pu



\def\squigbij{\;\; \diagxyto/<~>/<300> \;\;}

Here's an example of what they produce:

$%\resizebox{!}{60pt}{$
   \tcg{(P,A),Q} \squigbij \zha{O_A(P),J}
 % $}
$

\bsk

Even though these modules are not useful to other people some {\sl
  ideas} in them may be. (TO DO: give examples!)





%  ____
% | __ )  __ _ _ __ _ __
% |  _ \ / _` | '__| '__|
% | |_) | (_| | |  | |
% |____/ \__,_|_|  |_|
%
% «barr» (to ".barr")
\section{Versions of diagxy}
\label{diagxy}

Diagxy comes in two versions: the original one, \co{diagxy.tex}, that
is loaded with \co{\\input diagxy}, and another one, \co{xybarr.tex},
that is a module of xypic and is loaded with
\co{\\usepackage[barr,pdf]\{xy\}}. If you have a recent texlive,
installed with, say,

\url{https://www.tug.org/texlive/quickinstall.html}

\noindent then you should have these .tex files and their
documentation files at places like these:

% (find-angg ".emacs" "diagxy" "2018:")
% (find-tl2018sh "find * | sort | grep xy | grep barr")
% (find-tl2018sh "find * | sort | grep xy | grep dia")

\begin{verbatim}
/usr/local/texlive/2018/texmf-dist/tex/generic/barr/diagxy.tex
/usr/local/texlive/2018/texmf-dist/doc/generic/barr/diaxydoc.pdf
/usr/local/texlive/2018/texmf-dist/tex/generic/xypic/xybarr.tex
/usr/local/texlive/2018/texmf-dist/doc/generic/xypic/barrdoc.pdf
\end{verbatim}

You can access their docs online at these URLs:

{\footnotesize

\url{http://www.math.mcgill.ca/barr/papers/diaxydoc.pdf}

\url{http://tug.ctan.org/tex-archive/macros/latex/contrib/xypic/doc/barrdoc.pdf}

}

\msk

It seems that loading diagxy with \co{\\usepackage[barr,pdf]\{xy\}} is
incompatible with lualatex, and thus with Dednat6. Until we get that
fixed please use \co{\\input diagxy} instead.



%     _       __                                           _           
%    / \     / _| _____      __  ___  __ _ _ __ ___  _ __ | | ___  ___ 
%   / _ \   | |_ / _ \ \ /\ / / / __|/ _` | '_ ` _ \| '_ \| |/ _ \/ __|
%  / ___ \  |  _|  __/\ V  V /  \__ \ (_| | | | | | | |_) | |  __/\__ \
% /_/   \_\ |_|  \___| \_/\_/   |___/\__,_|_| |_| |_| .__/|_|\___||___/
%                                                   |_|                
%
% «a-few-samples» (to ".a-few-samples")
\section{``A few samples''}

The file \co{barrdoc.pdf} has a section called ``A few samples''
(5.9), that shows how to produce a certain big diagram in two ways;
one, that we will not discuss, using \qco{\\morphism}, and another
one, that we will call ``Barr's diagram'', that uses \qco{\\node} and
\qco{\\arrow} instead. In the next two pages we will compare Barr's
diagram to an approximate translation of it to Dednat6.

Note: I don't quite understand Barr's code for the two outermost
curved arrows --- it {\sl seems} that he resorts to low-level \Xy Pic
code in the ``modifier'' part of the shape parameter to create splined
arrows, but I couldn't figure out the exact meaning of
\qqco{@`\{c,(3000,0),(2700,2800),p\}} in the \Xy Pic reference
manual...

\newpage

%V % Source code for Barr's diagram:
%V %
%V $$\bfig
%V   \def\f{\bar f}
%V   \def\g{\bar g}
%V   \def\h{\bar h}
%V   \let\t\tau
%V   \node A11(0,2800)[(\h(\g\f))\t_A]
%V   \node A13(1200,2800)[((\h\g)\f)\t_A]
%V   \node A21(0,2400)[\h((\g\f)\t_A)]
%V   \node A22(600,2400)[\h(\g\f\t_A)]
%V   \node A23(1200,2400)[(\h\g(\f\t_A))]
%V   \node A32(600,2000)[\h(\g(\t_Bf))]
%V   \node A33(1200,2000)[(\h\g)(\t_Bf)]
%V   \node A34(1800,2000)[((\h\g)\t_B)f]
%V   \node A42(600,1600)[\h((\g\t_B)f)]
%V   \node A44(1800,1600)[(\h(\g\t_B))f]
%V   \node A52(600,1200)[\h((\t_C)g)f]
%V   \node A54(1800,1200)[(\h(\t_Cg))f]
%V   \node A62(600,800)[\h(\t_C(gf))]
%V   \node A63(1200,800)[(\h\t_C)(gf)]
%V   \node A64(1800,800)[\h(\t_C(gf))]
%V   \node A73(1200,400)[(\t_Dh)(gf)]
%V   \node A74(1800,400)[((\t_D)h)g]
%V   \node A75(2400,400)[(\t_D(hg))f]
%V   \node A83(1200,0)[\t_D(h(gf))]
%V   \node A85(2400,0)[\t_D((hg)f)]
%V   \arrow[A11`A13;]
%V   \arrow[A21`A11;]
%V   \arrow[A21`A22;]
%V   \arrow[A22`A23;]
%V   \arrow[A23`A13;]
%V   \arrow[A32`A22;\h(\g\t_f)]
%V   \arrow[A32`A33;]
%V   \arrow[A33`A23;(\h\g)\t_f]
%V   \arrow[A33`A34;]
%V   \arrow[A42`A44;]
%V   \arrow[A42`A32;]
%V   \arrow[A44`A34;]
%V   \arrow[A52`A42;\h(\t_gf)]
%V   \arrow[A52`A54;]
%V   \arrow[A54`A44;(\h\t_g)f]
%V   \arrow[A62`A52;]
%V   \arrow[A62`A63;]
%V   \arrow[A63`A64;]
%V   \arrow[A73`A63;\t_h(gf)]
%V   \arrow[A73`A74;]
%V   \arrow[A74`A64;\t_{(hg)f}]
%V   \arrow[A74`A75;]
%V   \arrow[A83`A73;]
%V   \arrow[A83`A85;]
%V   \arrow[A85`A75;]
%V   \arrow|r|/{@{>}@/_15pt/}/[A75`A34;\t_{hg}f]
%V   \arrow|l|/{@{>}@/^15pt/}/[A62`A21;\h(\t_C(gf))]
%V   \arrow|l|/{@{>}@`{c,(3000,0),(2700,2800),p}}/[A85`A13;\t_{hg}f]
%V   \arrow|r|/{@{>}@`{c,(-300,0),(-600,2400),p}}/[A83`A11;\t_{h(fg)}]
%V   \efig
%V $$
%L
%L verbdef "barrorig"


%V % Source code for its translation to Dednat6:
%V %
%V %D diagram barr-dednat6
%V %D 2Dx     100    +40    +40    +40    +40
%V %D 2D  100 A11 --------> A13
%V %D 2D       ^             ^
%V %D 2D       |             |
%V %D 2D  +27 A21 -> A22 -> A23
%V %D 2D       ^      ^      ^
%V %D 2D       |      |      |
%V %D 2D  +27  |     A32 -> A33 -> A34
%V %D 2D       |      ^             ^  ^
%V %D 2D       |      |             |   \
%V %D 2D  +27  |     A42 --------> A44   \
%V %D 2D       |      ^             ^     \
%V %D 2D        \     |             |      |
%V %D 2D  +27    \   A52 --------> A54     |
%V %D 2D          \   ^                    |
%V %D 2D           \  |                    |
%V %D 2D  +27        A62 -> A63 -> A64     |
%V %D 2D                     ^      ^      |
%V %D 2D                     |      |      |
%V %D 2D  +27               A73 -> A74 -> A75
%V %D 2D                     ^             ^
%V %D 2D                     |             |
%V %D 2D  +27               A83 --------> A85
%V %D 2D
%V %D ren A11     A13   ==>   (\h(\g\f))\t_A              ((\h\g)\f)\t_A
%V %D ren A21 A22 A23   ==>   \h((\g\f)\t_A) \h(\g\f\t_A) (\h\g(\f\t_A))
%V %D ren     A32 A33 A34     ==>            \h(\g(\t_Bf)) (\h\g)(\t_Bf) ((\h\g)\t_B)f
%V %D ren     A42     A44     ==>            \h((\g\t_B)f)               (\h(\g\t_B))f
%V %D ren     A52     A54     ==>            \h((\t_C)g)f                (\h(\t_Cg))f
%V %D ren     A62 A63 A64     ==>            \h(\t_C(gf)) (\h\t_C)(gf) \h(\t_C(gf))
%V %D ren         A73 A74 A75 ==>                          (\t_Dh)(gf) ((\t_D)h)g (\t_D(hg))f
%V %D ren         A83     A85 ==>                          \t_D(h(gf))            \t_D((hg)f)
%V %D
%V %D (( # Horizontal arrows:
%V %D    A11 A13 ->
%V %D    A21 A22 -> A22 A23 ->
%V %D    A32 A33 -> A33 A34 ->
%V %D    A42 A44 ->
%V %D    A52 A54 ->
%V %D    A62 A63 -> A63 A64 ->
%V %D    A73 A74 -> A74 A75 ->
%V %D    A83 A85 ->
%V %D
%V %D    # Simple vertical arrows:
%V %D    A11 A21 <-                       A13 A23 <-
%V %D    A22 A32 <- .plabel= r \h(\g\t_f) A23 A33 <- .plabel= r (\h\g)\t_f
%V %D    A32 A42 <-                       A34 A44 <-
%V %D    A42 A52 <- .plabel= r \h(\t_gf)  A44 A54 <- .plabel= r (\h\t_g)f
%V %D    A52 A62 <-
%V %D    A63 A73 <- .plabel= r \t_h(gf)   A64 A74 <- .plabel= r \t_{(hg)f}
%V %D    A73 A83 <- A75 A85 <-
%V %D
%V %D    # Curved vertical arrows:
%V %D    A75 A34 -> .curve= _15pt .plabel= r \t_{hg}f
%V %D    A62 A21 -> .curve= ^15pt .plabel= l \h(\t_C(gf))
%V %D    A83 A11 -> .mod= @`{c,(-300,-2835),(-800,-100),p} .plabel= r \t_{h(fg)}
%V %D    A85 A13 -> .mod= @`{c,(3000,-2000),(2700,-500),p} .plabel= l \t_{hg}f
%V %D
%V %D ))
%V %D enddiagram
%V %D
%V $$\pu
%V   \def\f{\bar f}
%V   \def\g{\bar g}
%V   \def\h{\bar h}
%V   \let\t\tau
%V   \diag{barr-dednat6}
%V $$
%L
%L verbdef "barrdednatsix"

{
\pu
\def\B#1{\resizebox{!}{6in}{#1}}
\hskip-8em
\B{\barrorig}
\;\;
\B{\barrdednatsix}
}

\newpage

% (find-es "diagxy" "shape")
% (find-barrdocpage 33 "5.9   A few samples")
% (find-barrdoctext 33 "5.9   A few samples")



% (find-angg ".emacs.papers" "xypic")
% (find-xyreferpage 12 "Figure 3: hobjectis")
% (find-xyrefertext 12 "Figure 3: hobjectis")
% (find-xyreferpage 22 "show control points")
% (find-xyrefertext 22 "show control points")
% (find-xyreferpage 39 "Figure 14: hpathis")
% (find-xyrefertext 39 "Figure 14: hpathis")
% (find-xyreferpage 21 "control points")
% (find-xyrefertext 21 "control points")

Barr's diagram:

\msk

\resizebox{!}{1.5in}{
$\bfig
  \def\f{\bar f}
  \def\g{\bar g}
  \def\h{\bar h}
  \let\t\tau
  \node A11(0,2800)[(\h(\g\f))\t_A]
  \node A13(1200,2800)[((\h\g)\f)\t_A]
  \node A21(0,2400)[\h((\g\f)\t_A)]
  \node A22(600,2400)[\h(\g\f\t_A)]
  \node A23(1200,2400)[(\h\g(\f\t_A))]
  \node A32(600,2000)[\h(\g(\t_Bf))]
  \node A33(1200,2000)[(\h\g)(\t_Bf)]
  \node A34(1800,2000)[((\h\g)\t_B)f]
  \node A42(600,1600)[\h((\g\t_B)f)]
  \node A44(1800,1600)[(\h(\g\t_B))f]
  \node A52(600,1200)[\h((\t_C)g)f]
  \node A54(1800,1200)[(\h(\t_Cg))f]
  \node A62(600,800)[\h(\t_C(gf))]
  \node A63(1200,800)[(\h\t_C)(gf)]
  \node A64(1800,800)[\h(\t_C(gf))]
  \node A73(1200,400)[(\t_Dh)(gf)]
  \node A74(1800,400)[((\t_D)h)g]
  \node A75(2400,400)[(\t_D(hg))f]
  \node A83(1200,0)[\t_D(h(gf))]
  \node A85(2400,0)[\t_D((hg)f)]
  \arrow[A11`A13;]
  \arrow[A21`A11;]
  \arrow[A21`A22;]
  \arrow[A22`A23;]
  \arrow[A23`A13;]
  \arrow[A32`A22;\h(\g\t_f)]
  \arrow[A32`A33;]
  \arrow[A33`A23;(\h\g)\t_f]
  \arrow[A33`A34;]
  \arrow[A42`A44;]
  \arrow[A42`A32;]
  \arrow[A44`A34;]
  \arrow[A52`A42;\h(\t_gf)]
  \arrow[A52`A54;]
  \arrow[A54`A44;(\h\t_g)f]
  \arrow[A62`A52;]
  \arrow[A62`A63;]
  \arrow[A63`A64;]
  \arrow[A73`A63;\t_h(gf)]
  \arrow[A73`A74;]
  \arrow[A74`A64;\t_{(hg)f}]
  \arrow[A74`A75;]
  \arrow[A83`A73;]
  \arrow[A83`A85;]
  \arrow[A85`A75;]
  \arrow|r|/{@{>}@/_15pt/}/[A75`A34;\t_{hg}f]
  \arrow|l|/{@{>}@/^15pt/}/[A62`A21;\h(\t_C(gf))]
  \arrow|l|/{@{>}@`{c,(3000,0),(2700,2800),p}}/[A85`A13;\t_{hg}f]
  \arrow|r|/{@{>}@`{c,(-300,0),(-600,2400),p}}/[A83`A11;\t_{h(fg)}]
  \efig
$}


\bsk
\bsk

%D diagram barr-dednat6
%D 2Dx     100    +40    +40    +40    +40
%D 2D  100 A11 --------> A13
%D 2D       ^             ^
%D 2D       |             |
%D 2D  +27 A21 -> A22 -> A23
%D 2D       ^      ^      ^
%D 2D       |      |      |
%D 2D  +27  |     A32 -> A33 -> A34
%D 2D       |      ^             ^  ^
%D 2D       |      |             |   \
%D 2D  +27  |     A42 --------> A44   \
%D 2D       |      ^             ^     \
%D 2D        \     |             |      |
%D 2D  +27    \   A52 --------> A54     |
%D 2D          \   ^                    |
%D 2D           \  |                    |
%D 2D  +27        A62 -> A63 -> A64     |
%D 2D                     ^      ^      |
%D 2D                     |      |      |
%D 2D  +27               A73 -> A74 -> A75
%D 2D                     ^             ^
%D 2D                     |             |
%D 2D  +27               A83 --------> A85
%D 2D
%D ren A11     A13   ==>   (\h(\g\f))\t_A              ((\h\g)\f)\t_A
%D ren A21 A22 A23   ==>   \h((\g\f)\t_A) \h(\g\f\t_A) (\h\g(\f\t_A))
%D ren     A32 A33 A34     ==>            \h(\g(\t_Bf)) (\h\g)(\t_Bf) ((\h\g)\t_B)f
%D ren     A42     A44     ==>            \h((\g\t_B)f)               (\h(\g\t_B))f
%D ren     A52     A54     ==>            \h((\t_C)g)f                (\h(\t_Cg))f
%D ren     A62 A63 A64     ==>            \h(\t_C(gf)) (\h\t_C)(gf) \h(\t_C(gf))
%D ren         A73 A74 A75 ==>                          (\t_Dh)(gf) ((\t_D)h)g (\t_D(hg))f
%D ren         A83     A85 ==>                          \t_D(h(gf))            \t_D((hg)f)
%D
%D (( # Horizontal arrows:
%D    A11 A13 ->
%D    A21 A22 -> A22 A23 ->
%D    A32 A33 -> A33 A34 ->
%D    A42 A44 ->
%D    A52 A54 ->
%D    A62 A63 -> A63 A64 ->
%D    A73 A74 -> A74 A75 ->
%D    A83 A85 ->
%D
%D    # Simple vertical arrows:
%D    A11 A21 <-                       A13 A23 <-
%D    A22 A32 <- .plabel= r \h(\g\t_f) A23 A33 <- .plabel= r (\h\g)\t_f
%D    A32 A42 <-                       A34 A44 <-
%D    A42 A52 <- .plabel= r \h(\t_gf)  A44 A54 <- .plabel= r (\h\t_g)f
%D    A52 A62 <-
%D    A63 A73 <- .plabel= r \t_h(gf)   A64 A74 <- .plabel= r \t_{(hg)f}
%D    A73 A83 <- A75 A85 <-
%D
%D    # Curved vertical arrows:
%D    A75 A34 -> .curve= _15pt .plabel= r \t_{hg}f
%D    A62 A21 -> .curve= ^15pt .plabel= l \h(\t_C(gf))
%D    A83 A11 -> .mod= @`{c,(-300,-2835),(-800,-100),p} .plabel= r \t_{h(fg)}
%D    A85 A13 -> .mod= @`{c,(3000,-2000),(2700,-500),p} .plabel= l \t_{hg}f
%D
%D ))
%D enddiagram
%D

My approximate translation of it to dednat6:

\msk

\resizebox{!}{1.5in}{
$\pu
  \def\f{\bar f}
  \def\g{\bar g}
  \def\h{\bar h}
  \let\t\tau
  \diag{barr-dednat6}
$
}


\newpage

%  ____            _ 
% |  _ \ ___ _ __ | |
% | |_) / _ \ '_ \| |
% |  _ <  __/ |_) | |
% |_| \_\___| .__/|_|
%           |_|      
%
% «repl» (to ".repl")
\section{The REPL}
% (turp 5)

Section 5 of the TUGBoat article --- called ``A read-eval-print-loop
(REPL)'' --- describes a way to start a Lua REPL in the middle of the
compilation of a \co{.tex} file. The .zip file for dednat6 includes a
file \co{2018dednat6-repl.tex} that lets you play with the REPL by
running just this:
%
$$\text{\co{lualatex 2018dednat6-repl.tex}}$$

The REPL can be used to inspect \qco{\\def}s, among other things. See
the ongoing discussing here:
%
$$\text{\url{http://tug.org/pipermail/luatex/2019-January/007034.html}}$$



%  ____             _                        _     
% | __ )  __ _  ___| | __      ___ _ __   __| |___ 
% |  _ \ / _` |/ __| |/ /____ / _ \ '_ \ / _` / __|
% | |_) | (_| | (__|   <_____|  __/ | | | (_| \__ \
% |____/ \__,_|\___|_|\_\     \___|_| |_|\__,_|___/
%                                                  
% «other-back-ends» (to ".other-back-ends")
\section{Other back-ends}

Dednat6 can be seen as a front-end for \co{proof.sty} and for (the
\qco{\\morphism} flavor of) \co{diagxy.tex}. It shouldn't be hard to
make it generate, say, code for Tikz instead of for diagxy, and code
for Sam Buss's \co{bussproofs.sty} instead of for \co{proof.sty}. If
you are interested in this, and you know Tikz/bussproofs/whatever
enough to give me examples of how the output should look, get in
touch!

{\sl Update:} of all other possible back-ends for 2D diagrams the
easiest one to implement seems to be the \qco{\\node}s and
\qco{\\arrow}s flavor of diagxy --- I guess it would be just a few
hours of work. If you think that this would be useful to you, drop me
a line!







\end{document}

% Local Variables:
% coding: utf-8-unix
% End:
