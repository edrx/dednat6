% A minimal test file for Dednat6 - with a diagram for Abilio Rodrigues
% This file: http://angg.twu.net/dednat6/demo-minimal-abilio.tex.html
%            http://angg.twu.net/dednat6/demo-minimal-abilio.tex
%                         (find-dednat6 "demo-minimal-abilio.tex")
%    Output: http://angg.twu.net/dednat6/demo-minimal-abilio.pdf
%
%       See: http://angg.twu.net/dednat6.html
%            http://angg.twu.net/dednat6.html#quick-start
%
%
% To compile (i.e., tex) this file, do this:
%
%   lualatex demo-minimal-abilio.tex
%
% this makes TeX and Lua process this file "in parallel", or sort of.
% Lua processes the trees and diagrams in 2D ascii art in comments
% that are in the lines that start with "%:" or "%D", converts them to
% TeX code, and sends that code to TeX.
%
% The section 3 ("3. Semi-preprocessors") of the TUGBoat article about
% dednat6 explains what this "in parallel (or sort of)" means. The PDF
% for the article is here:
%
%   https://tug.org/TUGboat/tb39-3/tb123ochs-dednat.pdf
%   http://angg.twu.net/dednat6/2018tugboat-rev2.pdf
% 
% Section 3 ends with this paragraph:
%
%   `\pu' means "process until" - or, more precisely, _make dednat6
%   process everything until this point that it hasn't processed yet_.
%   The first \pu [in the example] processes the lines 1--26 of
%   foo.tex, and "outputs" - i.e., sends to TeX - the first
%   \defded{my-tree}; the second \pu processes the lines 28--34 of
%   foo.tex, and `outputs' the second \defded{my-tree}. Thus, it is
%   not technically true that TeX and dednat6 process foo.tex in
%   parallel; dednat6 goes later, and each \pu is a synchronization
%   point.
%
%
% This comment block contains some emacs/eev-isms.
% You don't need to understand them.
% See: (find-eev-quick-intro "7.4. Commands with very short names")
%      http://angg.twu.net/eev-intros/find-eev-quick-intro.html#7.4
%
% (defun c () (interactive) (find-dednat6sh "lualatex -record demo-minimal-abilio.tex"))
% (defun d () (interactive) (find-pdf-page "~/dednat6/demo-minimal-abilio.pdf"))
% (defun e () (interactive) (find-dednat6 "demo-minimal-abilio.tex"))
% (defun u () (interactive) (find-latex-upload-links "demo-minimal-abilio"))
%   (find-pdf-page "~/dednat6/demo-minimal-abilio.pdf")
% http://angg.twu.net/dednat6/demo-minimal-abilio.pdf
%
\documentclass[oneside]{article}
  \usepackage{proof}   % For derivation trees ("%:" lines)
  \input diagxy        % For 2D diagrams ("%D" lines)
  \xyoption{curve}     % For the ".curve=" feature in 2D diagrams
\begin{document}
  \catcode`\^^J=10                      % (find-es "luatex" "spurious-omega")
  \directlua{dofile "dednat6load.lua"}  % (find-dednat6 "dednat6load.lua")


\title{Dednat6: a minimal test file (for Abilio)}
\author{Eduardo Ochs}
\maketitle

Some trees:
%
%L addabbrevs("->", "\\to ")
%
%:  [x]^1  f     [a]^1  a->b                                 H
%:  --------     -----------                                ...
%:     f(x)   g       b       b->c      A  B  C   E  F      \Pi
%:     --------       ------------      =======r  ::::\phi  ...
%:     g(f(x))             c               D       G         I
%:   ----------1         ----1             -------------------
%:   λx.g(f(x))          a->c                      J
%:
%:   ^x.g(f(x))          ^x.g(f(x)).t              ^bars
%:
\pu
$$\ded{x.g(f(x))} \qquad \ded{x.g(f(x)).t} \qquad \ded{bars}$$




\def\catA{\mathbf{A}}
\def\catB{\mathbf{B}}

A diagram:
%
%D diagram adj
%D 2Dx     100     +25     +25   +25
%D 2D  100 LA <--| A
%D 2D      |       |
%D 2D      |  <->  |
%D 2D      v       v
%D 2D  +25 B |--> RB       C --> D
%D 2D
%D 2D  +15 \catB \catA
%D 2D
%D (( LA A <-|
%D    LA B -> A RB ->
%D    B RB |->
%D    LA RB harrownodes nil 20 nil <->
%D    \catB \catA <- sl^ .plabel= a L
%D    \catB \catA -> sl_ .plabel= b R
%D
%D    C D -> .curve= ^10pt .plabel= a a
%D    C D -> .curve= _10pt .plabel= b b
%D ))
%D enddiagram
%D
$$\pu
  \diag{adj}
$$

\bigskip

A Kripke model (for Abilio Rodrigues):

%L forths["-"] = function () pusharrow("-") end

%D diagram Kripke-star
%D 2Dx     100     +30   +20
%D 2D  100               W_3
%D 2D                  /
%D 2D  +20 W_1 --- W_2
%D 2D                  \
%D 2D  +20               W_4
%D 2D
%D (( W_2 W_1 -
%D    W_2 W_3 -
%D    W_2 W_4 -
%D    
%D ))
%D enddiagram
%D
$$\pu
  \diag{Kripke-star}
$$



\end{document}

% Local Variables:
% coding: utf-8-unix
% End:
