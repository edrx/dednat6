% (find-angg "dednat6/extra-features.tex")
% (defun c () (interactive) (find-dednat6sh "lualatex -record extra-features.tex" :end))
% (defun d () (interactive) (find-pdf-page "~/dednat6/extra-features.pdf"))
% (defun e () (interactive) (find-dednat6 "extra-features.tex"))
% (defun u () (interactive) (find-latex-upload-links "extra-features"))
% (find-pdf-page "~/dednat6/extra-features.pdf")
% (find-sh0 "cp -v  ~/dednat6/extra-features.pdf /tmp/")
% (find-sh0 "cp -v  ~/dednat6/extra-features.pdf /tmp/pen/")
%   file:///home/edrx/dednat6/extra-features.pdf
%               file:///tmp/extra-features.pdf
%           file:///tmp/pen/extra-features.pdf
% http://angg.twu.net/dednat6/extra-features.pdf

% «.tree-bars»		(to "tree-bars")
% «.abbrevs»		(to "abbrevs")
% «.renaming»		(to "renaming")
% «.arrow-modifiers»	(to "arrow-modifiers")

\documentclass[oneside]{article}
\usepackage[colorlinks]{hyperref} % (find-es "tex" "hyperref")
\usepackage{amsmath}
\usepackage{amsfonts}
\usepackage{amssymb}
\usepackage{pict2e}
\usepackage[x11names,svgnames]{xcolor} % (find-es "tex" "xcolor")
\usepackage{color}                % (find-dednat6 "edrx15.sty" "colors")
\usepackage{colorweb}             % (find-es "tex" "colorweb")
%\usepackage{tikz}
%
% (find-dn6 "preamble6.lua" "preamble0")
\usepackage{proof}   % For derivation trees ("%:" lines)
\input diagxy        % For 2D diagrams ("%D" lines)
\xyoption{curve}     % For the ".curve=" feature in 2D diagrams
%
\begin{document}

\catcode`\^^J=10
\directlua{dofile "dednat6load.lua"}  % (find-LATEX "dednat6load.lua")

% \co: a low-level way to typeset code; a poor man's "\verb"
\def\co#1{{%
  \def\%{\char37}%
  \def\\{\char92}%
  \def\^{\char94}%
  \def\~{\char126}%
  \tt#1%
  }}
\def\qco#1{`\co{#1}'}
\def\qqco#1{``\co{#1}''}

\directlua{dofile "myverbatim.lua"} % (find-dednat6 "myverbatim.lua")
\def\myhbox#1#2#3{\setbox0=\hbox{#3}\ht0=#1\dp0=#2\box0}
\def\verbahbox#1{\hbox{\tt#1}}
\def\verbahbox#1{\myhbox{7pt}{2pt}{{\tt#1}}}
\def\bgbox#1{\bgcolorhbox{YellowOrangeLight}{#1}}
\def\bgcolorhbox#1#2{{%
  \setbox0\hbox{#2}%
  \setbox0\vbox{\vskip\fboxsep\box0\vskip\fboxsep}%
  \setbox0\hbox{\kern\fboxsep\box0\kern\fboxsep}%
  {\color{#1}{\smashedvrule{\wd0}{\ht0}{\dp0}}}%
  \box0%
  }}

\def\bsk{\bigskip}
\def\msk{\medskip}
\def\ssk{\smallskip}
\def\orange#1{{\color{orange}#1}}
\def\yellow#1{{\color{yellow}#1}}
\def\smashedvrule#1#2#3{\vrule width#1 height#2 depth#3 \kern-#1}
\def\bicolorhbox#1{{%
  \setbox0\hbox{#1}%
  \yellow{\smashedvrule{\wd0}{\ht0}{0pt}}%
  \orange{\smashedvrule{\wd0}{0pt}{\dp0}}%
  \box0%
  }}
\def\bhbox{\bicolorhbox}
\def\cded #1{\begin{matrix}\ded {#1}\end{matrix}}

\def\myvcenter#1{\begin{matrix}#1\end{matrix}}
\catcode`\^^O=13 \def{{\color{red}*}}

% (find-angg "LATEX/edrxchars.tex")
\catcode`Γ=13 \defΓ{\Gamma}
\catcode`∨=13 \def∨{\lor}
\catcode`⊢=13 \def⊢{\vdash}




%  _____                _                    
% |_   _| __ ___  ___  | |__   __ _ _ __ ___ 
%   | || '__/ _ \/ _ \ | '_ \ / _` | '__/ __|
%   | || | |  __/  __/ | |_) | (_| | |  \__ \
%   |_||_|  \___|\___| |_.__/ \__,_|_|  |___/
%                                            
% «tree-bars»  (to ".tree-bars")

\section{Other inference bars}

% (find-es "tex" "proof")
% (find-LATEXfile "2010hyps.tex" "\\DeduceSym")

All the examples of deduction trees in the TUGBoat article use
`\co{-}'s for the inference bars in the ASCII art representation. If
we use `\co{=}'s instead of `\co{-}'s we get double bars, and if we
use `\co{:}'s we get a line of vertical dots instead of a bar:
%:
%:                  Γ    Γ  [P]^1  Γ\;\;[Q]^1
%:                  :    ::::::::  ::::::::::
%:  Γ,P⊢R  Γ,Q⊢R    P∨Q    R           R
%:  ============    --------------------1
%:    Γ,P∨Q⊢R               R
%:
%:    ^or-1                 ^or-2
%:
$$\pu \ded{or-1} \quad\Longrightarrow\quad \ded{or-2}$$

You can change the number of vertical dots by redefining the macro
`\co{\\DeduceSym}'. For example:

\msk

%V \makeatletter
%V % Original with 4 dots (from proof.sty):
%V % \def\DeduceSym{\vtop{\baselineskip4\p@ \lineskiplimit\z@
%V %     \vbox{\hbox{.}\hbox{.}\hbox{.}}\hbox{.}}}
%V % New, with 3 dots:
%V \def\DeduceSym{\vtop{\baselineskip4\p@ \lineskiplimit\z@
%V     \vbox{\hbox{.}\hbox{.}}\hbox{.}}}
%V \makeatother
%
%L verbdef "foo"
%
$\pu\foo$





%     _    _     _
%    / \  | |__ | |__  _ __ _____   _____
%   / _ \ | '_ \| '_ \| '__/ _ \ \ / / __|
%  / ___ \| |_) | |_) | | |  __/\ V /\__ \
% /_/   \_\_.__/|_.__/|_|  \___| \_/ |___/
%
% «abbrevs»  (to ".abbrevs")
% (find-dn6file "dednat6.lua" "abbrevs")
% (find-dn6 "abbrevs.lua")

\section{Abbrevs}

The first Dednats did not support UTF-8, and the way to write a tree
node that would display as `$a→b$' was to write it as `\co{a->b}'
after running \co{addabbrevs("->", "\\to ")}. The module
\co{abbrevs.lua} implements this, and \co{unabbrev(str)} parses
\co{str} from left to right, at each point looking for the longest
string starting at that point that is an abbrev and replacing it by
its expansion, or leaving that character untouched if it doesn't have
an expansion. Here is an example:

%V %L addabbrevs("->", "\\to ")
%V %:
%V %:  [a]^1  a->b
%V %:  -----------
%V %:       b       b->c
%V %:       ------------
%V %:            c
%V %:          ----1
%V %:          a->c
%V %:
%V %:          ^a->c
%V %:
%V $$\pu \ded{a->c}$$
%L
%L verbdef "abbrevcode"

%L addabbrevs("->", "\\to ")
%:
%:  [a]^1  a->b
%:  -----------
%:       b       b->c
%:       ------------
%:            c
%:          ----1
%:          a->c
%:
%:          ^a->c
%:
%$$\pu \ded{a->c}$$

$$\pu
  \myvcenter{\abbrevcode}
  %\abbrevcode
  \qquad
  \to
  \qquad
  \cded{a->c}
$$

% (turp 2)

Abbrevs are also used in 2D diagrams, in a more complex way. Section
2.2 of the TUGBoat article explains how the grid words create a table
\co{nodes}, but it doesn't explain how the fields \co{.tex} and
\co{.TeX} in a node affect how it is displayed. The code below creates
nodes whose {\sl tags} are \co{"A"}, \co{"B"}, \co{"C"}, \co{"D"}, and
then changes the fields \co{.tex} and \co{.TeX} in some of these
nodes. The \TeX{} code for each node is calculated by the function
\co{node\_to\_TeX}, that expects a node (a table) and returns a
string. If \co{node\_to\_TeX} receives a node that has a \co{.TeX}
field then it returns that field unchanged, surrounded by
`\co{\{\}}'s; if it doesn't have a \co{.TeX} field but it has a
\co{.tex} field then it returns the result of running \co{unabbrev} on
that field and surrounding it with `\co{\{\}}'s; otherwise it returns
the result of running \co{unabbrev} on the tag surrounding it with
`\co{\{\}}'s. For example:

%V %D diagram nodes-and-abbrevs
%V %D 2Dx     100 +40
%V %D 2D  100 A -> B
%V %D 2D      |    |
%V %D 2D      v    v
%V %D 2D  +30 C -> D
%V %D 2D
%V %D (( B .tex= (a->b)
%V %D    C              .TeX= (a->b)
%V %D    D .tex= (a->b) .TeX= (a->b)
%V %L print("nodes:"); print(nodes)
%V %L print("A:", node_to_TeX(nodes["A"]))
%V %L print("B:", node_to_TeX(nodes["B"]))
%V %L print("C:", node_to_TeX(nodes["C"]))
%V %L print("D:", node_to_TeX(nodes["D"]))
%V %D    A B -> A C -> B D -> C D ->
%V %D ))
%V %D enddiagram
%L
%L verbdef "abbrevdiag"

%D diagram nodes-and-abbrevs
%D 2Dx     100  +40
%D 2D  100 A -> B
%D 2D      |    |
%D 2D      v    v
%D 2D  +30 C -> D
%D 2D
%D (( B .tex= (a->b)
%D    C              .TeX= (a->b)
%D    D .tex= (a->b) .TeX= (a->b)
%L print("nodes:"); print(nodes)
%L print("A:", node_to_TeX(nodes["A"]))
%L print("B:", node_to_TeX(nodes["B"]))
%L print("C:", node_to_TeX(nodes["C"]))
%L print("D:", node_to_TeX(nodes["D"]))
%D    A B -> A C -> B D -> C D ->
%D ))
%D enddiagram
%D
$$\pu
  \myvcenter{\abbrevdiag}
  \quad
  \to
  \quad
  \diag{nodes-and-abbrevs}
$$

The output of the \co{print()}s is:

{\footnotesize
\begin{verbatim}
nodes:
{   1={"noden"=1, "tag"="A", "x"=100, "y"=100},
    2={"noden"=2, "tag"="B", "x"=140, "y"=100, "tex"="(a->b)"},
    3={"noden"=3, "tag"="C", "x"=100, "y"=130,                 "TeX"="(a->b)"},
    4={"noden"=4, "tag"="D", "x"=140, "y"=130, "tex"="(a->b)", "TeX"="(a->b)"},
  "A"={"noden"=1, "tag"="A", "x"=100, "y"=100},
  "B"={"noden"=2, "tag"="B", "x"=140, "y"=100, "tex"="(a->b)"},
  "C"={"noden"=3, "tag"="C", "x"=100, "y"=130,                 "TeX"="(a->b)"},
  "D"={"noden"=4, "tag"="D", "x"=140, "y"=130, "tex"="(a->b)", "TeX"="(a->b)"}
}
A:      {A}
B:      {(a\to b)}
C:      {(a->b)}
D:      {(a->b)}
\end{verbatim}
}


%  ____                            _
% |  _ \ ___ _ __   __ _ _ __ ___ (_)_ __   __ _
% | |_) / _ \ '_ \ / _` | '_ ` _ \| | '_ \ / _` |
% |  _ <  __/ | | | (_| | | | | | | | | | | (_| |
% |_| \_\___|_| |_|\__,_|_| |_| |_|_|_| |_|\__, |
%                                          |___/
% «renaming»  (to ".renaming")
% (find-dn6 "diagforth.lua" "dxyren")
\section{Renaming}

The word \co{ren} in the language for 2D diagrams eats the rest of the
line, splits it at the `\co{==>}', and splits the material before the
`\co{==>}' into a list of {\sl tags}, $A_1, \ldots, A_n$, and the
material after `\co{==>}' into a list of {\sl texs}, $B_1, \ldots,
B_n$; these two lists must have the same length, and then \co{ren}
runs \co{nodes[$A_i$].tex = $B_i$} for each $i$ in $1,\ldots,n$. For
example:

%V %D diagram ren
%V %D 2Dx     100    +30
%V %D 2D  100 A1 <-| A2
%V %D 2D      |       |
%V %D 2D      v       v
%V %D 2D  +30 A3 |-> A4
%V %D 2D
%V %D ren A1 A2 ==> A{\times}B A
%V %D ren A3 A4 ==> C B{\to}C
%V %D
%V %D (( A1 A2 <-| .plabel= a ({\times}B)
%V %D    A1 A3 -> A2 A4 ->
%V %D    A3 A4 |-> .plabel= b (B{\to})
%V %D ))
%V %D enddiagram
%L
%L verbdef "diagren"

%D diagram ren
%D 2Dx     100    +30
%D 2D  100 A1 <-| A2
%D 2D      |       |
%D 2D      v       v
%D 2D  +30 A3 |-> A4
%D 2D
%D ren A1 A2 ==> A{\times}B A
%D ren A3 A4 ==> C B{\to}C
%D
%D (( A1 A2 <-| .plabel= a ({\times}B)
%D    A1 A3 -> A2 A4 ->
%D    A3 A4 |-> .plabel= b (B{\to})
%D ))
%D enddiagram
%D
$$\pu
  \myvcenter{\diagren}
  \quad
  \to
  \quad
  \diag{ren}
$$



%                        _       
%    _ __ ___   ___   __| |_____ 
%   | '_ ` _ \ / _ \ / _` |_____|
%  _| | | | | | (_) | (_| |_____|
% (_)_| |_| |_|\___/ \__,_|      
%                                
% «arrow-modifiers»  (to ".arrow-modifiers")
\section{Arrow modifiers}

% (find-es "diagxy" "shape")
% (find-dn6 "diagtex.lua" "arrow_to_TeX")
% (find-dn6 "diagtex.lua" "arrow_to_TeX" "arrow_to_TeX_pshL =")

The language for 2D diagrams in dednat6 has some words for curving and
sliding arrows:
%
%V %D diagram curve-slide
%V %D 2Dx     100 +40
%V %D 2D  100 A   B
%V %D 2D
%V %D 2D  +40 D   C
%V %D 2D
%V %D (( A B ->               .plabel= b \text{one}
%V %D    A B -> .slide=  5pt  .plabel= a \text{two}
%V %D    A B -> .slide= 20pt  .plabel= a \text{three}
%V %D    B C -> .curve= _10pt .plabel= l \text{four}
%V %D    B C -> .curve=  ^5pt .plabel= r \text{five}
%V %D    C D ->
%V %D    C D -> .curve=  _5pt
%V %D    C D -> .curve=  _5pt .slide= -5pt
%V %D           .plabel= a \text{six}
%V %D ))
%V %D enddiagram
%L
%L verbdef "curveslide"
%
%D diagram curve-slide
%D 2Dx     100 +40
%D 2D  100 A   B
%D 2D
%D 2D  +40 D   C
%D 2D
%D (( A B ->               .plabel= b \text{one}
%D    A B -> .slide=  5pt  .plabel= a \text{two}
%D    A B -> .slide= 20pt  .plabel= a \text{three}
%D    B C -> .curve= _10pt .plabel= l \text{four}
%D    B C -> .curve=  ^5pt .plabel= r \text{five}
%D    C D ->
%D    C D -> .curve=  _5pt
%D    C D -> .curve=  _5pt .slide= -5pt
%D           .plabel= a \text{six}
%D ))
%D enddiagram
%D
$$\pu
  \myvcenter{\footnotesize\curveslide}
  \quad
  \to
  \quad
  \diag{curve-slide}
$$

% (find-dn6 "diagforth.lua" "arrows")
% (find-dn6 "diagforth.lua" "arrows" "sl^^")

The words \qco{sl\^\^}, \qco{sl\^}, \qco{sl\_}, and \qco{sl\_\_} are
abbreviations for \qqco{.slide= 5pt}, \qqco{.slide= 2.5pt},
\qqco{.slide= -2.5pt}, \qqco{.slide= -5pt} respectively.

%L forths[".mod="] = function ()
%L     ds:pick(0).modifier = getword() or error()
%L   end

% %D diagram ??
% %D 2Dx     100 +40
% %D 2D  100 A   B
% %D 2D
% %D 2D  +40 D   C
% %D 2D
% %D (( A B -> .plabel= a 1 .mod= @<-5pt>@/_5pt/
% %D
% %D ))
% %D enddiagram
% %D
% $$\pu
%   \diag{??}
% $$









\end{document}

% Local Variables:
% coding: utf-8-unix
% End:
