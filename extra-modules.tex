% (find-angg "dednat6/extra-modules.tex")
% (defun c () (interactive) (find-dednat6sh "lualatex -record extra-modules.tex"))
% (defun d () (interactive) (find-pdf-page "~/dednat6/extra-modules.pdf"))
% (defun e () (interactive) (find-dednat6 "extra-modules.tex"))
% (defun u () (interactive) (find-latex-upload-links "extra-modules"))
% (find-pdf-page "~/dednat6/extra-modules.pdf")
% (find-sh0 "cp -v  ~/dednat6/extra-modules.pdf /tmp/")
% (find-sh0 "cp -v  ~/dednat6/extra-modules.pdf /tmp/pen/")
%   file:///home/edrx/dednat6/extra-modules.pdf
%               file:///tmp/extra-modules.pdf
%           file:///tmp/pen/extra-modules.pdf
% http://angg.twu.net/dednat6/extra-modules.pdf
%
% «.TCGs»		(to "TCGs")
% «.underbrace2d»	(to "underbrace2d")
% «.luarects»		(to "luarects")
%
\documentclass[oneside]{article}
\usepackage[colorlinks]{hyperref} % (find-es "tex" "hyperref")
\usepackage{amsmath}
\usepackage{amsfonts}
\usepackage{amssymb}
\usepackage{pict2e}
\usepackage[x11names,svgnames]{xcolor} % (find-es "tex" "xcolor")
%
% (find-dn6 "preamble6.lua" "preamble0")
\usepackage{proof}   % For derivation trees ("%:" lines)
\input diagxy        % For 2D diagrams ("%D" lines)
\xyoption{curve}     % For the ".curve=" feature in 2D diagrams
%
\begin{document}

\catcode`\^^J=10
\directlua{dofile "dednat6load.lua"}  % (find-LATEX "dednat6load.lua")

% \co: a low-level way to typeset code; a poor man's "\verb"
\def\co#1{{%
  \def\%{\char37}%
  \def\\{\char92}%
  \def\^{\char94}%
  \def\~{\char126}%
  \tt#1%
  }}
\def\qco#1{`\co{#1}'}
\def\qqco#1{``\co{#1}''}

\def\bsk{\bigskip}
\def\msk{\medskip}
\def\ssk{\smallskip}

% (find-LATEX "2017planar-has-defs.tex" "defzha-and-deftcg")
\def\defzha#1#2{\expandafter\def\csname zha-#1\endcsname{#2}}
\def\ifzhaundefined#1{\expandafter\ifx\csname zha-#1\endcsname\relax}
\def\zha#1{\ifzhaundefined{#1}
    \errmessage{UNDEFINED ZHA: #1}
  \else
    \csname zha-#1\endcsname
  \fi
}
\def\deftcg#1#2{\expandafter\def\csname tcg-#1\endcsname{#2}}
\def\iftcgundefined#1{\expandafter\ifx\csname tcg-#1\endcsname\relax}
\def\tcg#1{\iftcgundefined{#1}
    \errmessage{UNDEFINED TCG: #1}
  \else
    \csname tcg-#1\endcsname
  \fi
}
\def\defub#1#2{\expandafter\def\csname ub-#1\endcsname{#2}}
\def\ifubundefined#1{\expandafter\ifx\csname ub-#1\endcsname\relax}
\def\ub#1{\ifubundefined{#1}
    \errmessage{UNDEFINED UB: #1}
  \else
    \csname ub-#1\endcsname
  \fi
}
\def\und#1#2{\underbrace{#1}_{#2}}




\title{Dednat6: extra modules}

\author{Eduardo Ochs}

\maketitle





% (find-dn6 "dednat6.lua")
% (find-dn6 "dednat6.lua" "requires" "picture")
% (find-dn6 "picture.lua")
% (find-dn6 "zhas.lua")
% (find-dn6 "zhas.lua" "MixedPicture")

The code of Dednat6 --- inside the directory \co{dednat6/} --- is made
of several \co{.lua} files that are {\sl all} loaded by
\co{dednat6.lua}; there is no provision yet for loading only the
modules that are used in a given \co{.tex} file. This means that some
modules that are only useful to the author of Dednat6 (Eduardo Ochs,
a.k.a. ``me'') are always loaded.

Most of these extra modules were written to handle the objects
described in my series of papers about ``Planar Heyting Algebras'',
at:

\url{http://angg.twu.net/math-b.html#zhas-for-children-2}

\msk


%  _____ ____ ____     
% |_   _/ ___/ ___|___ 
%   | || |  | |  _/ __|
%   | || |__| |_| \__ \
%   |_| \____\____|___/
%                      
% «TCGs»  (to ".TCGs")

\section{Picture, zhas, zhaspecs, tcgs}


% (find-LATEXfile "2019ebl-five-appls.tex" "A strange J-operator")
%
%L tdims = TCGDims {qrh=5, q=15, crh=12, h=60, v=25, crv=7}   -- with v arrows
%L tspec_PA  = TCGSpec.new("46; 11 22 34 45, 25")
%L tspec_PAQ = TCGSpec.new("46; 11 22 34 45, 25", ".???", "???.?.")
%L tspec_PA :mp  ({zdef="O_A(P)"})  :addlrs():print()            :output()
%L tspec_PAQ:mp  ({zdef="O_A(P),J"}):addlrs():print()            :output()
%L tspec_PA :tcgq({tdef="(P,A)",   meta="1pt p"}, "lr q h v ap") :output()
%L tspec_PAQ:tcgq({tdef="(P,A),Q", meta="1pt p"}, "lr q h v ap") :output()
%L
%L tspec_PAC = TCGSpec.new("46; 11 22 34 45, 25", "?...", "???...")
%L tspec_PAC:mp  ({zdef="closed-op"}) :addlrs():print()            :output()
%L tspec_PAC:tcgq({tdef="closed-op", meta="1pt p"}, "lr q h v ap") :output()
%L 
\pu

\def\squigbij{\;\; \diagxyto/<~>/<300> \;\;}

Here's an example of what they produce:

$%\resizebox{!}{60pt}{$
   \tcg{(P,A),Q} \squigbij \zha{O_A(P),J}
 % $}
$

\bsk

Even though these modules are not useful to other people some {\sl
  ideas} in them may be. (TO DO: give examples!)



%  _   _           _           _                        
% | | | |_ __   __| | ___ _ __| |__  _ __ __ _  ___ ___ 
% | | | | '_ \ / _` |/ _ \ '__| '_ \| '__/ _` |/ __/ _ \
% | |_| | | | | (_| |  __/ |  | |_) | | | (_| | (_|  __/
%  \___/|_| |_|\__,_|\___|_|  |_.__/|_|  \__,_|\___\___|
%                                                       
% «underbrace2d»  (to ".underbrace2d")
% (find-LATEX "2019seminario-hermann.tex" "values-of-subexpressions-S4")

\section{Underbrace2d}


%UB  P \lor Q \to P \lor Q
%UB  -      -     -      -
%UB  0      1     0      1
%UB  --------     --------
%UB      1            0
%UB  ---------------------
%UB            0
%L
%L defub "PvQ -> PaQ"
$$\pu \ub{PvQ -> PaQ}
$$



%  _                              _       
% | |   _   _  __ _ _ __ ___  ___| |_ ___ 
% | |  | | | |/ _` | '__/ _ \/ __| __/ __|
% | |__| |_| | (_| | | |  __/ (__| |_\__ \
% |_____\__,_|\__,_|_|  \___|\___|\__|___/
%                                         
% «luarects»  (to ".luarects")
% (find-LATEX "2019seminario-hermann.tex" "lattice-non-planar")

\section{Luarects}


% (find-dn6 "luarects.lua")





\end{document}

% Local Variables:
% coding: utf-8-unix
% End:
